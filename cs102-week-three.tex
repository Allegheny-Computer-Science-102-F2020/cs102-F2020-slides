\documentclass[14pt,aspectratio=169]{beamer}

\usepackage{pgfpages}
\usepackage{fancyvrb}
\usepackage{pgfplots}

\usepackage{minted}
\usemintedstyle{tango}

\usepackage{amsfonts}

\usepackage{moresize}
\usepackage{anyfontsize}

\usepackage{tikz}
\usetikzlibrary{arrows,shapes}
\usetikzlibrary{arrows.meta}

\tikzstyle{process}=[rectangle, draw, thick, text width=5em, text centered, minimum height=2.5em, fill=gray!40]
\tikzstyle{entity}=[rounded rectangle, draw, thick, text width=5em, text centered, minimum height=1.5em, fill=gray!40]

\usetheme{auriga}
\usecolortheme{auriga}

\setbeamercolor{background canvas}{bg=lightgray}

% define some colors for a consistent theme across slides
\definecolor{red}{RGB}{181, 23, 0}
\definecolor{blue}{RGB}{0, 118, 186}
\definecolor{gray}{RGB}{146, 146, 146}

\title{Discrete Structures: \\ Computational and Mathematical Tasks in Python}

\author{{\bf Gregory M. Kapfhammer}}

\institute[shortinst]{{\bf Department of Computer Science, Allegheny College}}

\begin{document}

{
  \setbeamercolor{page number in head/foot}{fg=background canvas.bg}
  \begin{frame}
    \titlepage
  \end{frame}
}

%% Slide
%
\begin{frame}{Technical Question}
  %
  \begin{center}
    %
    {\large How do I use iteration and conditional logic in a Python program to
      perform computational tasks like processing the contents of a file and
      mathematical tasks like using Newton's method to approximate the square root
    of a number?}
    %
  \end{center}
  %
  \vspace{2ex}
  %
  \begin{center}
    %
    \small Let's learn how to use the Python programming language to implement
    programs that perform useful computational and mathematical tasks!
    %
  \end{center}
  %
\end{frame}

% Slide
%
\begin{frame}{Python Programming Retrospective}
  %
  \begin{itemize}
    %
    \item Intuitively read the code segments to grasp their behavior
      %
      \vspace*{-.15in}
      %
    \item Key components of the Python programming segments
      %
      \begin{itemize}
        %
        \item Function calls
          %
        \item Assignment statements
          %
        \item Iteration constructs
          %
        \item Conditional logic
          %
        \item Variable creation
          %
        \item Variable computations
          %
        \item Variable output
          %
      \end{itemize}
      %
      \vspace*{-.2in}
      %
    \item Investigate the syntax and semantics of these components!
      %
  \end{itemize}
  %
\end{frame}

% Slide
%
\begin{frame}[fragile]
  \frametitle{Python Programs are Sequences of Statements}
  \normalsize
  \hspace*{-.65in}
  \begin{minipage}{6in}
    \vspace*{.25in}
    \begin{minted}[mathescape, numbersep=5pt, fontsize=\large]{python}
    file = open("emails")
    for line in file:
      name, email = line.split(",")
      if name = "John Davis":
        print(email)
    \end{minted}
  \end{minipage}
  \vspace*{.25in}
  \begin{center}
    %
    \normalsize \noindent A Python program is a sequence of statements \\
    \normalsize \noindent Programs contain both {\em simple} and {\em compound} statements \\
    \normalsize \noindent Why is this, technically, not a ``Python program''? \\
    %
  \end{center}
  %
\end{frame}

% Slide
%
\begin{frame}[fragile]
  \frametitle{Simple and Compound Statements in Python}
  \hspace*{-.6in}
  \begin{minipage}{6in}
    \begin{minted}[mathescape, numbersep=5pt, fontsize=\large]{python}
    sum = 0
    count = 0
    file = open("observations")
    for line in file:
      n = int(line)
      sum += n
      count += 1
    print(sum/count)
    \end{minted}
  \end{minipage}
\end{frame}

% Slide
%
\begin{frame}{Industry-Standard Python Programming}
  %
  \begin{itemize}
    %
    \item Please use Python 3 for all of your programs!
      %
      \vspace*{-.15in}
      %
    \item Add ``docstring'' comments to your Python programs
      %
      \begin{itemize}
        %
        \item Module
          %
        \item Class
          %
        \item Function
          %
      \end{itemize}
      %
      \vspace*{-.2in}
      %
    \item Add comments for important blocks of your program
      %
      \vspace*{-.2in}
      %
    \item Use descriptive variable and function names
      %
      \vspace*{-.2in}
      %
    \item The book does not always adhere to industry standards!
      %
  \end{itemize}
  %
\end{frame}

% Slide
%
\begin{frame}[fragile]
  \frametitle{Command-Line Interfaces for Python Programs}
  \normalsize
  \hspace*{-.15in}
  \begin{minipage}{6in}
    \vspace*{.15in}
    \begin{minted}[mathescape, numbersep=5pt, fontsize=\footnotesize]{python}
def main(
    a: float = typer.Option(1),
    b: float = typer.Option(2),
    c: float = typer.Option(2)
):
    """Calculate roots of a quadratic eqn with quadratic formula."""
    typer.echo(f"Calculating the roots of a quadratic equation with:")
    typer.echo(f"   a = {a}")
    typer.echo(f"   b = {b}")
    typer.echo(f"   c = {c}")
    x_one, x_two = rootfind.calculate_quadratic_equation_roots(a, b, c)
    typer.echo(f"   x_one = {x_one}")
    typer.echo(f"   x_two = {x_two}")

if __name__ == "__main__":
    typer.run(main)
    \end{minted}
  \end{minipage}
  %
  \vspace*{.05in}
  %
  \begin{center}
    %
    \normalsize \noindent This program can accept user input through the command line \\
    %
  \end{center}
  %
\end{frame}

% Slide
%
\begin{frame}[fragile]
  \frametitle{Calculating the Roots of a Quadratic Function}
  \hspace*{-.1in}
  \begin{minipage}{6in}
    %
    \vspace*{.2in}
    %
    \begin{minted}[mathescape, numbersep=5pt, fontsize=\small]{python}
def calc_quad_eqn_roots(a: float, b: float, c: float):
    """Calculate the roots of a quadratic equation."""
    D = (b * b - 4 * a * c) ** 0.5
    x_one = (-b + D) / (2 * a)
    x_two = (-b - D) / (2 * a)
    return x_one, x_two
    \end{minted}
  \end{minipage}
  %
  \vspace*{.05in}
  %
  \begin{center}
    %
    \normalsize \noindent {\bf Input}: Three floating-point inputs called {\tt a}, {\tt b}, and {\tt c}\\
    \normalsize \noindent {\bf Output}: Two floating-point outputs called {\tt x\_one} and {\tt x\_two}\\
    \normalsize \noindent {\bf Behavior}: Calculate the roots of a quadratic equation\\
    \normalsize \noindent {\bf Question}: How does this function work? How do you test it?
    %
  \end{center}
  %
\end{frame}

% Slide
%
\begin{frame}[fragile]
  \frametitle{Testing the Root Finding Function}
  \hspace*{-.1in}
  \begin{minipage}{6in}
    %
    \vspace*{.2in}
    %
    \begin{minted}[mathescape, numbersep=5pt, fontsize=\small]{python}
def test_calculate_x_values_non_imaginary():
    """Check calculation of x values."""
    a = 1
    b = 2
    c = 1
    x_one, x_two = rootfind.calc_quad_eqn_roots(a, b, c)
    assert x_one == -1.0
    assert x_two == -1.0
    \end{minted}
  \end{minipage}
  %
  \vspace*{.05in}
  %
  \begin{center}
    %
    \normalsize \noindent {\bf Input}: Three floating-point inputs with values of 1, 2, and 1
    \normalsize \noindent {\bf Actual Output}: Stored in the variables called {\tt x\_one} and {\tt x\_two}\\
    \normalsize \noindent {\bf Assertions}: Ensure that the function calculated correctly\\
    %
  \end{center}
  %
\end{frame}

%% Slide
%
\begin{frame}{Mathematical Equations for Root Finding}
  %
  \vspace*{-.5in}
  %
  \begin{center}
    %
    \fontsize{20}{30}\selectfont
    %
    \begin{equation*}
      %
      x_1=\frac{-b+\sqrt{b^2-4ac}}{2a}
      %
    \end{equation*}
    %
    \begin{equation*}
      %
      x_2=\frac{-b-\sqrt{b^2-4ac}}{2a}
      %
    \end{equation*}
    %
  \end{center}
  %
  \vspace{.05ex}
  %
  \begin{center}
    %
    \small What is the meaning of the ``root'' of a quadratic equation? \\
    %
    \small For a quadratic equation $f(x)= a \times x^2 + b \times x +c$, the
    variables $x_1$ and $x_2$ are the points where the equation's output are
    the same\\
    %
  \end{center}
  %
\end{frame}

% Slide
%
\begin{frame}[fragile]
  \frametitle{Conditional Logic in Python Programs}
  \normalsize
  \hspace*{-.65in}
  \begin{minipage}{6in}
    \vspace*{.25in}
    \begin{minted}[mathescape, numbersep=5pt, fontsize=\large]{python}
    file = open("emails")
    for line in file:
      name, email = line.split(",")
      if name == "John Davis":
        print(email)
    \end{minted}
  \end{minipage}
  \vspace*{.25in}
  \begin{center}
    %
    \normalsize \noindent The {\tt if} statement allows the program to make a decision \\
    \normalsize \noindent The {\tt ==} operator compares {\tt name} to a String literal \\
    \normalsize \noindent When the condition is {\tt true} the {\tt print} executes \\
    %
  \end{center}
  %
\end{frame}

% Slide
%
\begin{frame}[fragile]
  \frametitle{Iteration Constructs in Python Programs}
  \normalsize
  \hspace*{-.65in}
  \begin{minipage}{6in}
    \vspace*{.25in}
    \begin{minted}[mathescape, numbersep=5pt, fontsize=\large]{python}
    file = open("emails")
    for line in file:
      name, email = line.split(",")
      if name == "John Davis":
        print(email)
    \end{minted}
  \end{minipage}
  \vspace*{.25in}
  \begin{center}
    %
    \normalsize \noindent The {\tt for} loop allows a Python program to repeat an operation \\
    \normalsize \noindent This {\tt for} loop ``iterates'' through all of the lines in the file \\
    \normalsize \noindent Indentation indicates which statements are in the {\tt for} loop \\
    %
  \end{center}
  %
\end{frame}

% Slide
%
\begin{frame}[fragile]
  \frametitle{Using {\tt for} Loops in Python Programs}
  \normalsize
  \hspace*{-.65in}
  \begin{minipage}{6in}
    \vspace*{.25in}
    \begin{minted}[mathescape, numbersep=5pt, fontsize=\large]{python}
    for i in range(20):
        print("2 to the " + str(i)
              + " power is " + str(2**i))
    \end{minted}
  \end{minipage}
  \vspace*{.25in}
  \begin{center}
    %
    \normalsize \noindent The {\tt for} loop displays the powers of 2 from 0 to 19 \\
    \normalsize \noindent The {\tt range} function returns the values from 0 to 19 \\
    \normalsize \noindent The {\tt 2**i} performs the computation $2^i$ \\
    \normalsize \noindent The {\tt str} function converts the integers to a string \\
    %
  \end{center}
  %
\end{frame}

% Slide
%
\begin{frame}[fragile]
  \frametitle{Using {\tt while} Loops in Python Programs}
  \normalsize
  \hspace*{-.65in}
  \begin{minipage}{6in}
    \vspace*{.25in}
    \begin{minted}[mathescape, numbersep=5pt, fontsize=\large]{python}
    i = 0
    while i < 20:
        print("2 to the " + str(i)
              + " power is " + str(2**i))
        i += 1
    \end{minted}
  \end{minipage}
  \vspace*{.1in}
  \begin{center}
    %
    \normalsize \noindent The {\tt for} loop and the {\tt while} loop are equivalent\\
    \normalsize \noindent The {\tt print} statement is the same for both of the loops\\
    \normalsize \noindent The purpose of {\tt i += 1} is to increment the loop counter\\
    %
  \end{center}
  %
\end{frame}

% Slide
%
\begin{frame}[fragile]
  \frametitle{Output of a Python Program Using Iteration}
  \normalsize
  \hspace*{-.35in}
  \begin{minipage}{6in}
    \vspace*{.25in}
    \begin{minted}[mathescape, numbersep=5pt, fontsize=\small]{text}
    2 to the 0 power is 1
    2 to the 1 power is 2
    2 to the 2 power is 4
    2 to the 3 power is 8
    2 to the 4 power is 16
    2 to the 5 power is 32
    2 to the 6 power is 64
    2 to the 7 power is 128
    2 to the 8 power is 256
    2 to the 9 power is 512
    \end{minted}
  \end{minipage}
  \vspace*{.05in}
  \begin{center}
    %
    \normalsize \noindent The output displays the counter and the computation\\
    %
  \end{center}
  %
\end{frame}

% Slide
%
\begin{frame}[fragile]
  \frametitle{Output of a Python Program Using Iteration}
  \normalsize
  \hspace*{-.35in}
  \begin{minipage}{6in}
    \vspace*{.25in}
    \begin{minted}[mathescape, numbersep=5pt, fontsize=\small]{text}
    2 to the 10 power is 1024
    2 to the 11 power is 2048
    2 to the 12 power is 4096
    2 to the 13 power is 8192
    2 to the 14 power is 16384
    2 to the 15 power is 32768
    2 to the 16 power is 65536
    2 to the 17 power is 131072
    2 to the 18 power is 262144
    2 to the 19 power is 524288
    \end{minted}
  \end{minipage}
  \vspace*{.05in}
  \begin{center}
    %
    \normalsize \noindent The {\tt for} and {\tt while} loops create the same output!\\
    %
  \end{center}
  %
\end{frame}

% Slide
%
\begin{frame}{Computing with Python Programs}
  %
  \begin{itemize}
    %
    \item Python programs can express many computations
      %
      \vspace*{-.15in}
      %
    \item Key components of the Python programming segments
      %
      \begin{itemize}
        %
        \item Function calls
          %
        \item Assignment statements
          %
        \item Iteration constructs
          %
        \item Conditional logic
          %
        \item Variable creation
          %
        \item Variable computations
          %
        \item Variable output
          %
      \end{itemize}
      %
      \vspace*{-.2in}
      %
    \item Programs must have correct format, comments, and tests!
      %
  \end{itemize}
  %
\end{frame}

\end{document}
