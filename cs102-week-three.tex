\documentclass[14pt,aspectratio=169]{beamer}

\usepackage{pgfpages}
\usepackage{fancyvrb}
\usepackage{pgfplots}

\usepackage{minted}
\usemintedstyle{tango}

\usepackage{amsfonts}

\usepackage{moresize}
\usepackage{anyfontsize}

\usepackage{tikz}
\usetikzlibrary{arrows,shapes}
\usetikzlibrary{arrows.meta}

\tikzstyle{process}=[rectangle, draw, thick, text width=5em, text centered, minimum height=2.5em, fill=gray!40]
\tikzstyle{entity}=[rounded rectangle, draw, thick, text width=5em, text centered, minimum height=1.5em, fill=gray!40]

\usetheme{auriga}
\usecolortheme{auriga}

\setbeamercolor{background canvas}{bg=lightgray}

% define some colors for a consistent theme across slides
\definecolor{red}{RGB}{181, 23, 0}
\definecolor{blue}{RGB}{0, 118, 186}
\definecolor{gray}{RGB}{146, 146, 146}

\title{Discrete Structures: \\ Basics of Mathematics and Programming}

\author{{\bf Gregory M. Kapfhammer}}

\institute[shortinst]{{\bf Department of Computer Science, Allegheny College}}

\begin{document}

{
  \setbeamercolor{page number in head/foot}{fg=background canvas.bg}
  \begin{frame}
    \titlepage
  \end{frame}
}

%% Slide
%
\begin{frame}{Technical Question}
  %
  \begin{center}
    %
    {\large How do I use iteration and conditional logic in a Python program to
      perform computational tasks like processing the contents of a file and
      mathematical tasks like using Newton's method to approximate the square root
    of a number?}
    %
  \end{center}
  %
  \vspace{2ex}
  %
  \begin{center}
    %
    \small Let's learn how to use the Python programming language to implement
    programs that perform useful computational and mathematical tasks!
    %
  \end{center}
  %
\end{frame}

% Slide
%
\begin{frame}{Python Programming Retrospective}
  %
  \begin{itemize}
    %
    \item Intuitively read the code segments to grasp their behavior
      %
      \vspace*{-.15in}
      %
    \item Key components of the Python programming segments
      %
      \begin{itemize}
        %
        \item Function calls
          %
        \item Assignment statements
          %
        \item Iteration constructs
          %
        \item Conditional logic
          %
        \item Variable creation
          %
        \item Variable computations
          %
        \item Variable output
          %
      \end{itemize}
      %
      \vspace*{-.2in}
      %
    \item Investigate the syntax and semantics of these components!
      %
  \end{itemize}
  %
\end{frame}

% Slide
%
\begin{frame}[fragile]
  \frametitle{Using Python to Find a Name in a File}
  \normalsize
  \hspace*{-.65in}
  \begin{minipage}{6in}
    \vspace*{.25in}
    \begin{minted}[mathescape, numbersep=5pt, fontsize=\large]{python}
    file = open("emails")
    for line in file:
      name, email = line.split(",")
      if name = "John Davis":
        print(email)
    \end{minted}
  \end{minipage}
  \vspace*{.25in}
  \begin{center}
    %
    \normalsize \noindent A Python program is a sequence of statements \\
    \normalsize \noindent Programs contain both {\em simple} and {\em compound} statements \\
    \normalsize \noindent Why is this, technically, not a ``Python program''? \\
    %
  \end{center}
  %
\end{frame}

% Slide
%
\begin{frame}[fragile]
  \frametitle{Using Python to Average Numerical Values}
  \hspace*{-.6in}
  \begin{minipage}{6in}
    \begin{minted}[mathescape, numbersep=5pt, fontsize=\large]{python}
    sum = 0
    count = 0
    file = open("observations")
    for line in file:
      n = int(line)
      sum += n
      count += 1
    print(sum/count)
    \end{minted}
  \end{minipage}
\end{frame}

% Slide
%
\begin{frame}{Enhancing Python Programs}
  %
  \begin{itemize}
    %
    \item Please use Python 3 for all of your programs!
      %
      \vspace*{-.15in}
      %
    \item Add ``docstring'' comments to your Python programs
      %
      \begin{itemize}
        %
        \item Module
          %
        \item Class
          %
        \item Function
          %
      \end{itemize}
      %
      \vspace*{-.2in}
      %
    \item Add comments for important blocks of your program
      %
      \vspace*{-.2in}
      %
    \item Use descriptive variable and function names
      %
      \vspace*{-.2in}
      %
    \item The book does not always adhere to industry standards!
      %
  \end{itemize}
  %
\end{frame}

\end{document}
