\documentclass[14pt,aspectratio=169]{beamer}

\usepackage{pgfpages}
\usepackage{fancyvrb}
\usepackage{pgfplots}

\usepackage{minted}
\usemintedstyle{tango}

\usepackage{amsfonts}

\usepackage{moresize}
\usepackage{anyfontsize}

\usepackage{tikz}
\usetikzlibrary{arrows,shapes}
\usetikzlibrary{arrows.meta}

\tikzstyle{process}=[rectangle, draw, thick, text width=5em, text centered, minimum height=2.5em, fill=gray!40]
\tikzstyle{entity}=[rounded rectangle, draw, thick, text width=5em, text centered, minimum height=1.5em, fill=gray!40]

\usetheme{auriga}
\usecolortheme{auriga}

\setbeamercolor{background canvas}{bg=lightgray}

% define some colors for a consistent theme across slides
\definecolor{red}{RGB}{181, 23, 0}
\definecolor{blue}{RGB}{0, 118, 186}
\definecolor{gray}{RGB}{146, 146, 146}

\title{Discrete Structures: \\ Functions and Expressions in the Python Language}

\author{{\bf Gregory M. Kapfhammer}}

\institute[shortinst]{{\bf Department of Computer Science, Allegheny College}}

\begin{document}

{
  \setbeamercolor{page number in head/foot}{fg=background canvas.bg}
  \begin{frame}
    \titlepage
  \end{frame}
}

%% Slide
%
\begin{frame}{Technical Question}
  %
  \begin{center}
    %
    {\large How do I use non-recursive functions, recursive functions, and
    lambda expressions to perform mathematical operations such as computing the
  absolute value of a number and the mean and median of a sequence of numbers?}
    %
  \end{center}
  %
  \vspace{2ex}
  %
  \begin{center}
    %
    \small Let's learn how to use the Python programming language to different
    types of functions that perform mathematical and statistical computations!
    %
  \end{center}
  %
\end{frame}

% Slide
%
\begin{frame}{Python Programming with Functions}
  %
  \begin{itemize}
    %
    \item Intuitively read the functions to grasp their behavior
      %
      \vspace*{-.15in}
      %
    \item Key components of the Python functions
      %
      \begin{itemize}
        %
        \item Definition of the function
          %
        \item Parameter(s) that serve as the input
          %
        \item Body that performs a computation
          %
        \item Function return value(s) that produce output
          %
        \item Invocation of the function
          %
        \item Collecting the output of the function
          %
        \item Test case(s) for the function
          %
      \end{itemize}
      %
      \vspace*{-.2in}
      %
    \item Investigate the ways to {\em define} and {\em call} Python functions!
      %
  \end{itemize}
  %
\end{frame}

% Slide
%
\begin{frame}[fragile]
  \frametitle{Computing the Absolute Value of a Number}
  \normalsize
  \hspace*{-.65in}
  \begin{minipage}{6in}
    \vspace*{.25in}
    \begin{minted}[mathescape, numbersep=5pt, fontsize=\large]{python}
    def abs(n):
        if n >= 0:
            return n
        else:
            return -n
    \end{minted}
  \end{minipage}
  \vspace*{.25in}
  \begin{center}
    %
    \normalsize \noindent The absolute value of a number is its distance from zero \\
    \normalsize \noindent What is the output of {\tt print(str(abs(10)))}? \\
    \normalsize \noindent What is the output of {\tt print(str(abs(-10)))}? \\
    %
  \end{center}
  %
\end{frame}

% Slide
%
\begin{frame}[fragile]
  \frametitle{Alternative Absolute Value Computation}
  \normalsize
  \hspace*{-.65in}
  \begin{minipage}{6in}
    \vspace*{.25in}
    \begin{minted}[mathescape, numbersep=5pt, fontsize=\large]{python}
    def abs(n):
        if n >= 0:
            return n
        return -n
    \end{minted}
  \end{minipage}
  \vspace*{.25in}
  \begin{center}
    %
    \normalsize \noindent Does this function compute the same value? \\
    \normalsize \noindent Which implementation of {\tt abs} do you prefer? \\
    \normalsize \noindent Which implementation of {\tt abs} does {\tt pylint} prefer? \\
    \normalsize \noindent There are different ways to implement the same function! \\
    %
  \end{center}
  %
\end{frame}

% Slide
%
\begin{frame}[fragile]
  \frametitle{Using Newton's Method in a Function}
  \hspace*{-.8in}
  \begin{minipage}{6in}
    \begin{minted}[mathescape, numbersep=5pt, fontsize=\large]{python}
    def sqrt(num: int, tol: float):
        guess = 1.0
        while abs(num - guess*guess) > tol:
            guess = guess -
            (guess*guess - num)/(2*guess)
        return guess
    \end{minted}
  \end{minipage}
  \vspace*{.05in}
  \begin{center}
    %
    \normalsize \noindent What are the benefits of defining this as a function? \\
    \normalsize \noindent What is the meaning of ``{\tt num:int}'' and ``{\tt tol:float}''? \\
    %
  \end{center}
\end{frame}

% Slide
%
\begin{frame}{Reminders About Industry-Standard Python}
  %
  \begin{itemize}
    %
    \item Please use Python 3 for all of your programs!
      %
      \vspace*{-.15in}
      %
    \item Add ``docstring'' comments to your Python programs
      %
      \begin{itemize}
        %
        \item Module
          %
        \item Class
          %
        \item Function
          %
      \end{itemize}
      %
      \vspace*{-.2in}
      %
    \item Add comments for important blocks of your function
      %
      \vspace*{-.2in}
      %
    \item Use descriptive variable and function names
      %
      \vspace*{-.2in}
      %
    \item Use type hints to explain the type of each parameter
      %
  \end{itemize}
  %
\end{frame}

% Slide
%
\begin{frame}[fragile]
  \frametitle{Recursive Functions in Python Programs}
  \hspace*{-.8in}
  \begin{minipage}{6in}
    \begin{minted}[mathescape, numbersep=5pt, fontsize=\large]{python}
    def factorial(number: int):
        if number == 1:
            return 1
        return number * factorial(number - 1)

    num = 5
    print("The factorial of " + str(num) +
           " is " + str(factorial(num)))
    \end{minted}
  \end{minipage}
  \vspace*{.05in}
\end{frame}

% Slide
%
\begin{frame}{Recursive Computation of the Factorial Function}
  %
  \begin{itemize}
    %
    \item As an equation: $n! = n \times n-1 \times n-2 \times \ldots \times 1$
      %
      \vspace*{-.15in}
      %
    \item What are the parts of a recursive function in Python?
      %
      \begin{itemize}
        %
        \item Defined by cases using conditional logic
          %
        \item A function definition that calls itself
          %
        \item A recursive call that makes progress to a base case
          %
        \item A base case that stops the recursive function calls
          %
      \end{itemize}
      %
      \vspace*{-.2in}
      %
    \item Repeatedly perform an operation through function calls
      %
      \vspace*{-.2in}
      %
    \item What would happen if you input a negative number?
      %
      \vspace*{-.2in}
      %
    \item How could you write this function with iteration?
      %
  \end{itemize}
  %
\end{frame}

% Slide
%
\begin{frame}[fragile]
  \frametitle{Finding the Parts of a Recursive Function}
  \hspace*{-.8in}
  \begin{minipage}{6in}
    \begin{minted}[mathescape, numbersep=5pt, fontsize=\large]{python}
    def factorial(number: int):
        if number == 1:
            return 1
        return number * factorial(number - 1)

    num = 5
    print("The factorial of " + str(num) +
           " is " + str(factorial(num)))
    \end{minted}
  \end{minipage}
  \vspace*{.05in}
\end{frame}

% Slide
%
\begin{frame}{Investigations in Python and Mathematics}
  %
  \begin{itemize}
    %
    \item How do you pick between the {\tt for} and {\tt while} loops?
      %
      \vspace*{-.15in}
      %
    \item Program for the root finding of a quadratic equation
      %
      \begin{itemize}
        %
        \item {\bf Q1}: What does it mean if a number is imaginary?
          %
        \item {\bf Q2}: What happens if the root of the equation is imaginary?
          %
        \item {\bf Q3}: How do tests use assertions for floating point values?
          %
        \item {\bf Q4}: How can you confirm that the function works correctly?
          %
        \item {\bf Q5}: How do you know when you have tested enough?
          %
      \end{itemize}
      %
      \vspace*{-.2in}
      %
    \item Can you translate the root finding equation into a Python program?
      Can you ensure its correctness? Can you follow industry standards for
      comments, format, and testing?
      %
  \end{itemize}
  %
\end{frame}

\end{document}
