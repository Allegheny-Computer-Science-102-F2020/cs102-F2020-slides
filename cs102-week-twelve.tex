\documentclass[14pt,aspectratio=169]{beamer}

\usepackage{pgfpages}
\usepackage{fancyvrb}
\usepackage{pgfplots}

\usepackage{minted}
\usemintedstyle{tango}

\usepackage{amsfonts}

\usepackage{moresize}
\usepackage{anyfontsize}

\usepackage{tikz}
\usetikzlibrary{arrows,shapes}
\usetikzlibrary{arrows.meta}

\tikzstyle{process}=[rectangle, draw, thick, text width=5em, text centered, minimum height=2.5em, fill=gray!40]
\tikzstyle{entity}=[rounded rectangle, draw, thick, text width=5em, text centered, minimum height=1.5em, fill=gray!40]

\usetheme{auriga}
\usecolortheme{auriga}

\setbeamercolor{background canvas}{bg=lightgray}

% define some colors for a consistent theme across slides

\definecolor{red}{RGB}{181, 23, 0}
\definecolor{blue}{RGB}{0, 118, 186}
\definecolor{gray}{RGB}{146, 146, 146}

\title{\vspace*{-.25in}Discrete Structures: \\ Using Relations\\ and Objects\\ in Python}

\author{{\bf Gregory M. Kapfhammer}}

\institute[shortinst]{{\bf Department of Computer Science, Allegheny College}}

\begin{document}

{
  \setbeamercolor{page number in head/foot}{fg=background canvas.bg}
  \begin{frame}
    \titlepage
  \end{frame}
}

%% Slide
%
\begin{frame}{Technical Question}
  %
  \hspace*{.25in}
  %
  \begin{minipage}{5in}
    %
    \vspace*{.1in}
    %
    \begin{center}
      %
      {\large How do I use the concept of a relation and the industrially
      relevant practice of object-oriented programming to correctly implement
      easy-to-understand programs in Python?}
      %
    \end{center}
    %
  \end{minipage}
  %
  \vspace{2ex}
  %
  \begin{center}
    %
    \small Let's learn how to implement and use relations and objects in the
    Python programming language! Please review the content from previous weeks
    so that you can keep this week's content in an appropriate context. Also,
    try out all of the Python code that is available in {\tt cs102-F2020-slides}!
    %
  \end{center}
  %
\end{frame}

% Slide
%
\begin{frame}{The Mathematical Concept of a Relation}
  %
  \begin{itemize}
    %
    \item As first introduced in Chapter 1, a relation is a connection between
      two or more entities (e.g., sets or ordered pairs)
      %
      \vspace*{-.15in}
      %
    \item Understanding relations in discrete mathematics:
      %
      \begin{itemize}
        %
        \item {\bf Binary relation}: set of ordered pairs
        %
        \item {\bf Ternary relation}: set of ordered triples
        %
        \item {\bf Quaternary relation}: set of ordered quadruples
        %
        \item {\bf n-ary relation}: set of ordered n-tuples
        %
      \end{itemize}
      %
      \vspace*{-.2in}
      %
    \item First, think about the number of related elements
      %
      \vspace*{-.2in}
      %
    \item Consider the relationship between the elements as well
      %
      \vspace*{-.2in}
      %
    \item How do I create and use relations in Python? Objects?
      %
  \end{itemize}
  %
\end{frame}

% Slide
%
\begin{frame}[fragile]
  \frametitle{Reviewing a Program that Uses Ordered Pairs}
  \hspace*{-.25in}
  \begin{minipage}{6in}
    \vspace*{.2in}
    \begin{minted}[mathescape, numbersep=5pt, fontsize=\normalsize]{python}
def factorial_iterative(number):
    factorial_list = []
    value = 1
    factorial_value = 1
    values = list(range(1, number + 1))
    for value in values:
        factorial_value = factorial_value * value
        factorial_list.
           append((value, factorial_value))
    return ((value, factorial_value),
           factorial_list)
    \end{minted}
  \end{minipage}
  \vspace*{.25in}
  \begin{center}
    %
    \normalsize \noindent What is the output of this program? \\
    \normalsize \noindent Are there other ways to create a {\tt Tuple[Tuple[int, int], List[Tuple[int, int]]]}? \\
    \normalsize \noindent What operations can we perform with a {\tt FiniteSet}? \\
    %
  \end{center}
  %
\end{frame}

% Slide
%
\begin{frame}{Understanding the Use of Ordered Pairs}
  %
  \begin{itemize}
    %
    \item {\bf Input}: {\tt number: int}
      %
      \vspace*{-.15in}
      %
    \item {\bf First Output}: {\tt Tuple[int, int]}
      %
      \vspace*{-.15in}
      %
    \item {\bf Second Output}: {\tt List[Tuple[int, int]]}
      %
      \vspace*{-.15in}
      %
    \item The two return values are also combined in a {\tt Tuple}!
      %
      \vspace*{-.15in}
      %
    \item Recall {\tt Tuple[int, int]} is Pythonic for an ordered pair
      %
      \vspace*{-.15in}
      %
    \item What is the need for such a complicated return value?
      %
      \vspace*{-.15in}
      %
    \item What is the best way to represent relationships in Python?
      %
  \end{itemize}
  %
\end{frame}

% Slide
%
\begin{frame}[fragile]
  \frametitle{Using a List to Represent Objects in Python}
  \hspace*{-.15in}
  \begin{minipage}{6in}
    \vspace*{.1in}
    \begin{minted}[mathescape, numbersep=5pt, fontsize=\normalsize]{python}
bosco = ["Bosco", 6, "Havanese"]
faith = ["Faith", 14, "Havanese"]

print(f"The name of the dog is: {bosco[0]}")
print(f"The age of the dog is: {bosco[1]}")
print(f"The breed of the dog is: {bosco[2]}")

print(f"The name of the dog is: {faith[0]}")
print(f"The age of the dog is: {faith[1]}")
print(f"The breed of the dog is: {faith[2]}")
    \end{minted}
  \end{minipage}
  \vspace*{.05in}
  \begin{center}
    %
    \normalsize \noindent What are the drawbacks of this approach? \\
    %
  \end{center}
  %
\end{frame}

% Slide
%
\begin{frame}[fragile]
  \frametitle{Output from Using a List to Represent Objects}
  \begin{minipage}{6in}
    \vspace*{.15in}
    \begin{minted}[mathescape, numbersep=5pt, fontsize=\normalsize]{text}
The name of the dog is: Bosco
The age of the dog is: 6
The breed of the dog is: Havanese

The name of the dog is: Faith
The age of the dog is: 14
The breed of the dog is: Havanese
    \end{minted}
  \end{minipage}
  \vspace*{.05in}
  \begin{center}
    %
    \normalsize \noindent How can this approach lead to defects? \\
    \normalsize \noindent Will this approach ensure object uniformity? \\
    \normalsize \noindent Are there other approaches to creating objects? \\
    %
  \end{center}
  %
\end{frame}

% Slide
%
\begin{frame}[fragile]
  \frametitle{Using a Dictionary to Represent Objects}
  \begin{minipage}{6in}
    \vspace*{.2in}
    \begin{minted}[mathescape, numbersep=5pt, fontsize=\normalsize]{python}
bosco = {}

bosco["Name"] = "Bosco"
bosco["Age"] = 6
bosco["Breed"] = "Havanese"

faith = {}

faith["Name"] = "Faith"
faith["Age"] = 14
faith["Breed"] = "Havanese"
    \end{minted}
  \end{minipage}
  \vspace*{.25in}
  \begin{center}
    %
    \normalsize \noindent What is the output of this program? \\
    \normalsize \noindent Are there other ways to create a {\tt Tuple[Tuple[int, int], List[Tuple[int, int]]]}? \\
    \normalsize \noindent What operations can we perform with a {\tt FiniteSet}? \\
    %
  \end{center}
  %
\end{frame}

% Slide
%
\begin{frame}[fragile]
  \frametitle{Accessing a Dictionary Representing an Object}
  \hspace*{-.1in}
  \begin{minipage}{6in}
    \vspace*{.2in}
    \begin{minted}[mathescape, numbersep=5pt, fontsize=\normalsize]{python}
print(f"The dog's name is: {bosco['Name']}")
print(f"The dog's age is: {bosco['Age']}")
print(f"The dog's breed is: {bosco['Breed']}")

print(f"The dog's name is: {faith['Name']}")
print(f"The dog's age is: {faith['Age']}")
print(f"The dog's breed is: {faith['Breed']}")
    \end{minted}
  \end{minipage}
  \vspace*{.1in}
  \begin{center}
    %
    \normalsize \noindent What is the output of this program? \\
    \normalsize \noindent Benefits compared to the list-based approach? \\
    \normalsize \noindent Differences compared to the list-based approach? \\
    %
  \end{center}
  %
\end{frame}

% Slide
%
\begin{frame}[fragile]
  \frametitle{Output from Using a Dictionary for Objects}
  \begin{minipage}{6in}
    \vspace*{.15in}
    \begin{minted}[mathescape, numbersep=5pt, fontsize=\normalsize]{text}
The dog's name is: Bosco
The dog's age is: 6
The dog's breed is: Havanese

The dog's name is: Faith
The dog's age is: 14
The dog's breed is: Havanese
    \end{minted}
  \end{minipage}
  \vspace*{.05in}
  \begin{center}
    %
    \normalsize \noindent How can this approach lead to defects? \\
    \normalsize \noindent Will this approach ensure object uniformity? \\
    \normalsize \noindent Are there other approaches to creating objects? \\
    %
  \end{center}
  %
\end{frame}

% Slide
%
\begin{frame}[fragile]
  \frametitle{Representing an Object with a Class Definition}
  \hspace*{-.1in}
  \begin{minipage}{6in}
    \vspace*{.2in}
    \begin{minted}[mathescape, numbersep=5pt, fontsize=\large]{python}
class Dog:
  # constructor for the Dog class
  def __init__(self, name, age, breed):
      self.name = name
      self.age = age
      self.breed = breed
    \end{minted}
  \end{minipage}
  \vspace*{.1in}
  \begin{center}
    %
    \normalsize \noindent What are the components of the {\tt Dog} class? \\
    \normalsize \noindent How do you create an instance of the {\tt Dog} class? \\
    %
  \end{center}
  %
\end{frame}

% Slide
%
\begin{frame}[fragile]
  \frametitle{Constructing an Instance of the {\tt Dog} Class}
  \hspace*{-.1in}
  \begin{minipage}{6in}
    \vspace*{.2in}
    \begin{minted}[mathescape, numbersep=5pt, fontsize=\normalsize]{python}
bosco = Dog("Bosco", 6, "Havanese")

print(bosco)

print(f"The dog's name is: {bosco.name}")
print(f"The dog's age is: {bosco.age}")
print(f"The dog's breed is: {bosco.breed}")
    \end{minted}
  \end{minipage}
  \vspace*{.1in}
  \begin{center}
    %
    \normalsize \noindent How is this different than the other approaches? \\
    \normalsize \noindent What is the output of this program? \\
    %
  \end{center}
  %
\end{frame}

% Slide
%
\begin{frame}[fragile]
  \frametitle{Constructing Another Instance of the {\tt Dog} Class}
  \hspace*{-.1in}
  \begin{minipage}{6in}
    \vspace*{.2in}
    \begin{minted}[mathescape, numbersep=5pt, fontsize=\normalsize]{python}
faith = Dog("Faith", 14, "Havanese")

print(faith)

print(f"The dog's name is: {faith.name}")
print(f"The dog's age is: {faith.age}")
print(f"The dog's breed is: {faith.breed}")
    \end{minted}
  \end{minipage}
  \vspace*{.1in}
  \begin{center}
    %
    \normalsize \noindent What is the output of this program? \\
    \normalsize \noindent Location of {\tt bosco} and {\tt faith} in computer's
    memory? \\
    %
  \end{center}
  %
\end{frame}

% Slide
%
\begin{frame}[fragile]
  \frametitle{Output from Using Objects in Python}
  \begin{minipage}{6in}
    \vspace*{.15in}
    \begin{minted}[mathescape, numbersep=5pt, fontsize=\large]{text}
<__main__.Dog object at 0x7f6090b44a90>

The dog's name is: Bosco
The dog's age is: 6
The dog's breed is: Havanese
    \end{minted}
  \end{minipage}
  \vspace*{.05in}
  \begin{center}
    %
    \normalsize \noindent {\tt bosco.name} accesses the dog's name \\
    \normalsize \noindent {\tt bosco.age} accesses the dog's age \\
    \normalsize \noindent {\tt bosco.breed} accesses the dog's breed \\
    %
  \end{center}
  %
\end{frame}

% Slide
%
\begin{frame}[fragile]
  \frametitle{Output from Using Objects in Python}
  \begin{minipage}{6in}
    \vspace*{.15in}
    \begin{minted}[mathescape, numbersep=5pt, fontsize=\large]{text}
<__main__.Dog object at 0x7f6090b44af0>

The dog's name is: Faith
The dog's age is: 14
The dog's breed is: Havanese
    \end{minted}
  \end{minipage}
  \vspace*{.05in}
  \begin{center}
    %
    \normalsize \noindent {\tt faith.name} accesses the dog's name \\
    \normalsize \noindent {\tt faith.age} accesses the dog's age \\
    \normalsize \noindent {\tt faith.breed} accesses the dog's breed \\
    %
  \end{center}
  %
\end{frame}

% Slide
%
\begin{frame}[fragile]
  \frametitle{Defining Methods for a Python Objects}
  \hspace*{-.15in}
  \begin{minipage}{6in}
    \vspace*{.2in}
    \begin{minted}[mathescape, numbersep=5pt, fontsize=\normalsize]{python}
# define a class to represent a Dog entity
class Dog:

    # define a description method for the dog
    def description(self):
        return f"{self.name} is a {self.age}
                 years old {self.breed}"

    # define an action method for the dog
    def action(self, action):
        return f"Hey, {self.name} {action}!"
    \end{minted}
  %
  \end{minipage}
  %
\end{frame}

% Slide
%
\begin{frame}{Creating Mathematical Mappings in Python}
  %
  \begin{itemize}
    %
    \item Dictionaries are super discrete structures with benefits!
      %
      \vspace*{-.2in}
      %
    \item Combine dictionaries and functions or use separately
      %
      \vspace*{-.2in}
      %
    \item Using dictionaries and functions in Python programs:
      %
      \begin{itemize}
        %
        \item {\bf Q1}: What are the characteristics of functions in Python?
          %
        \item {\bf Q2}: What are the characteristics of dictionaries in Python?
          %
        \item {\bf Q3}: How do you decide between a dictionary or a function?
          %
        \item {\bf Q4}: What functions can process a multiset discrete
          structure?
          %
        \item {\bf Q5}: How to combine functions and dictionaries for
          efficiency?
          %
      \end{itemize}
      %
      \vspace*{-.2in}
      %
    \item See \url{https://realpython.com/python3-object-oriented-programming/}
      for more on how to create and use objects in Python!
      %
  \end{itemize}
  %
\end{frame}

\end{document}
