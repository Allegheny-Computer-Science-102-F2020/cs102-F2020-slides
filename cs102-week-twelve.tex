\documentclass[14pt,aspectratio=169]{beamer}

\usepackage{pgfpages}
\usepackage{fancyvrb}
\usepackage{pgfplots}

\usepackage{minted}
\usemintedstyle{tango}

\usepackage{amsfonts}

\usepackage{moresize}
\usepackage{anyfontsize}

\usepackage{tikz}
\usetikzlibrary{arrows,shapes}
\usetikzlibrary{arrows.meta}

\tikzstyle{process}=[rectangle, draw, thick, text width=5em, text centered, minimum height=2.5em, fill=gray!40]
\tikzstyle{entity}=[rounded rectangle, draw, thick, text width=5em, text centered, minimum height=1.5em, fill=gray!40]

\usetheme{auriga}
\usecolortheme{auriga}

\setbeamercolor{background canvas}{bg=lightgray}

% define some colors for a consistent theme across slides

\definecolor{red}{RGB}{181, 23, 0}
\definecolor{blue}{RGB}{0, 118, 186}
\definecolor{gray}{RGB}{146, 146, 146}

\title{\vspace*{-.25in}Discrete Structures: \\ Using Relations\\ and Objects\\ in Python}

\author{{\bf Gregory M. Kapfhammer}}

\institute[shortinst]{{\bf Department of Computer Science, Allegheny College}}

\begin{document}

{
  \setbeamercolor{page number in head/foot}{fg=background canvas.bg}
  \begin{frame}
    \titlepage
  \end{frame}
}

%% Slide
%
\begin{frame}{Technical Question}
  %
  \hspace*{.25in}
  %
  \begin{minipage}{5in}
    %
    \vspace*{.1in}
    %
    \begin{center}
      %
      {\large How do I use the concept of a relation and the industrially
      relevant practice of object-oriented programming to correctly implement
      easy-to-understand programs in Python?}
      %
    \end{center}
    %
  \end{minipage}
  %
  \vspace{2ex}
  %
  \begin{center}
    %
    \small Let's learn how to implement and use relations and objects in the
    Python programming language! Please review the content from previous weeks
    so that you can keep this week's content in an appropriate context. Also,
    try out all of the Python code that is available in {\tt cs102-F2020-slides}!
    %
  \end{center}
  %
\end{frame}

% Slide
%
\begin{frame}{The Mathematical Concept of a Relation}
  %
  \begin{itemize}
    %
    \item As first introduced in Chapter 1, a relation is a connection between
      two or more entities (e.g., sets or ordered pairs)
      %
      \vspace*{-.15in}
      %
    \item Understanding relations in discrete mathematics:
      %
      \begin{itemize}
        %
        \item {\bf Binary relation}: set of ordered pairs
        %
        \item {\bf Ternary relation}: set of ordered triples
        %
        \item {\bf Quaternary relation}: set of ordered quadruples
        %
        \item {\bf n-ary relation}: set of ordered n-tuples
        %
      \end{itemize}
      %
      \vspace*{-.2in}
      %
    \item First, think about the number of related elements
      %
      \vspace*{-.2in}
      %
    \item Consider the relationship between the elements as well
      %
      \vspace*{-.2in}
      %
    \item How do I implement relations in Python? Objects?
      %
  \end{itemize}
  %
\end{frame}

% Slide
%
\begin{frame}[fragile]
  \frametitle{Reviewing a Program that Uses Ordered Pairs}
  \hspace*{-.25in}
  \begin{minipage}{6in}
    \vspace*{.2in}
    \begin{minted}[mathescape, numbersep=5pt, fontsize=\normalsize]{python}
def factorial_iterative(number):
    factorial_list = []
    value = 1
    factorial_value = 1
    values = list(range(1, number + 1))
    for value in values:
        factorial_value = factorial_value * value
        factorial_list.
           append((value, factorial_value))
    return ((value, factorial_value),
           factorial_list)
    \end{minted}
  \end{minipage}
  \vspace*{.25in}
  \begin{center}
    %
    \normalsize \noindent What is the output of this program? \\
    \normalsize \noindent Are there other ways to create a {\tt Tuple[Tuple[int, int], List[Tuple[int, int]]]}? \\
    \normalsize \noindent What operations can we perform with a {\tt FiniteSet}? \\
    %
  \end{center}
  %
\end{frame}

% Slide
%
\begin{frame}{Understanding the Use of Ordered Pairs}
  %
  \begin{itemize}
    %
    \item {\bf Input}: {\tt number: int}
      %
      \vspace*{-.15in}
      %
    \item {\bf First Output}: {\tt Tuple[int, int]}
      %
      \vspace*{-.15in}
      %
    \item {\bf Second Output}: {\tt List[Tuple[int, int]]}
      %
      \vspace*{-.15in}
      %
    \item The two return values are also combined in a {\tt Tuple}!
      %
      \vspace*{-.15in}
      %
    \item Recall {\tt Tuple[int, int]} is Pythonic for an ordered pair
      %
      \vspace*{-.15in}
      %
    \item What is the need for such a complicated return value?
      %
      \vspace*{-.15in}
      %
    \item Also, what is the meaning of {\tt factorial\_list.append}?
      %
  \end{itemize}
  %
\end{frame}

% Slide
%
\begin{frame}{Creating Mathematical Mappings in Python}
  %
  \begin{itemize}
    %
    \item Dictionaries are super discrete structures with benefits!
      %
      \vspace*{-.2in}
      %
    \item Combine dictionaries and functions or use separately
      %
      \vspace*{-.2in}
      %
    \item Using dictionaries and functions in Python programs:
      %
      \begin{itemize}
        %
        \item {\bf Q1}: What are the characteristics of functions in Python?
          %
        \item {\bf Q2}: What are the characteristics of dictionaries in Python?
          %
        \item {\bf Q3}: How do you decide between a dictionary or a function?
          %
        \item {\bf Q4}: What functions can process a multiset discrete
          structure?
          %
        \item {\bf Q5}: How to combine functions and dictionaries for
          efficiency?
          %
      \end{itemize}
      %
      \vspace*{-.2in}
      %
    \item See \url{https://realpython.com/python-dicts/} for more on how to
      create and use dictionaries in Python!
      %
  \end{itemize}
  %
\end{frame}

\end{document}
