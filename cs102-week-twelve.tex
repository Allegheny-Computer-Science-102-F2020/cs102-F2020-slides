\documentclass[14pt,aspectratio=169]{beamer}

\usepackage{pgfpages}
\usepackage{fancyvrb}
\usepackage{pgfplots}

\usepackage{minted}
\usemintedstyle{tango}

\usepackage{amsfonts}

\usepackage{moresize}
\usepackage{anyfontsize}

\usepackage{tikz}
\usetikzlibrary{arrows,shapes}
\usetikzlibrary{arrows.meta}

\tikzstyle{process}=[rectangle, draw, thick, text width=5em, text centered, minimum height=2.5em, fill=gray!40]
\tikzstyle{entity}=[rounded rectangle, draw, thick, text width=5em, text centered, minimum height=1.5em, fill=gray!40]

\usetheme{auriga}
\usecolortheme{auriga}

\setbeamercolor{background canvas}{bg=lightgray}

% define some colors for a consistent theme across slides

\definecolor{red}{RGB}{181, 23, 0}
\definecolor{blue}{RGB}{0, 118, 186}
\definecolor{gray}{RGB}{146, 146, 146}

\title{\vspace*{-.25in}Discrete Structures: \\ Using Relations\\ and Objects\\ in Python}

\author{{\bf Gregory M. Kapfhammer}}

\institute[shortinst]{{\bf Department of Computer Science, Allegheny College}}

\begin{document}

{
  \setbeamercolor{page number in head/foot}{fg=background canvas.bg}
  \begin{frame}
    \titlepage
  \end{frame}
}

%% Slide
%
\begin{frame}{Technical Question}
  %
  \hspace*{.25in}
  %
  \begin{minipage}{5in}
    %
    \vspace*{.1in}
    %
    \begin{center}
      %
      {\large How do I use the concept of a relation and the industrially
      relevant practice of object-oriented programming to correctly implement
      easy-to-understand programs in Python?}
      %
    \end{center}
    %
  \end{minipage}
  %
  \vspace{2ex}
  %
  \begin{center}
    %
    \small Let's learn how to implement and use relations and objects in the
    Python programming language! Please review the content from previous weeks
    so that you can keep this week's content in an appropriate context. Also,
    try out all of the Python code that is available in {\tt cs102-F2020-slides}!
    %
  \end{center}
  %
\end{frame}

% Slide
%
\begin{frame}{The Concept of a Mathematical Mapping}
  %
  \begin{itemize}
    %
    \item As first introduced in Chapter 1, a mapping is a set of ordered pairs
      in which no two have the same first element
      %
      \vspace*{-.15in}
      %
    \item Implementing functions in the Python language:
      %
      \begin{itemize}
        %
        \item Functions can accept zero or more inputs
        %
        \item Functions can produce zero or more outputs
        %
        \item Higher-order functions can accept and produce functions
        %
        \item One function's input can become another function's output
        %
        \item Imperative or declarative specification of functions
        %
        \item Discrete structures normally have a ``function interface''
        %
      \end{itemize}
      %
      \vspace*{-.2in}
      %
    \item It is possible to implement a function in many different ways! How do
      we decide which approach is the best?
      %
  \end{itemize}
  %
\end{frame}

% Slide
%
\begin{frame}[fragile]
  \frametitle{Inputs and Outputs with The Collatz Function}
  \hspace*{-.2in}
  \begin{minipage}{6in}
    \vspace*{.25in}
    \begin{minted}[mathescape, numbersep=5pt, fontsize=\large]{python}
def collatz(number: int) -> Iterator[int]:
    yield number
    while number != 1:
        if number % 2 == 0:
            number = number // 2
        else:
            number = 3 * number + 1
        yield number
    \end{minted}
  \end{minipage}
  \vspace*{.25in}
  \begin{center}
    %
    \normalsize \noindent What is the output of this program? \\
    \normalsize \noindent Are there other ways to create a {\tt FiniteSet}? \\
    \normalsize \noindent What operations can we perform with a {\tt FiniteSet}? \\
    %
  \end{center}
  %
\end{frame}

% Slide
%
\begin{frame}{Should You Pick a Dictionary or a Function?}
  %
  \begin{itemize}
    %
    \item Wait, are dictionaries and functions the same? Well ...
      %
      \vspace*{-.15in}
      %
    \item Picking between a discrete structure and a computation:
      %
      \begin{itemize}
        %
        \item A discrete structure must be finite and fit into memory
        %
        \item A function might be the only viable solution for many keys
        %
        \item A discrete structure can often be mutable, giving flexibility
        %
        \item It is often hard to make a function mutable at runtime
        %
        \item The lookup of a value for a key is done efficiently in a
          dictionary
        %
        \item A function that computes the value for a key may take time
        %
        \item But, what is going to be easier to program, test, and maintain?
        %
      \end{itemize}
      %
      \vspace*{-.2in}
      %
    \item The honest answer is ``it depends''! Use your judgement and
      benchmarks! Defend your decision with evidence!
      %
  \end{itemize}
  %
\end{frame}

% Slide
%
\begin{frame}{Creating Mathematical Mappings in Python}
  %
  \begin{itemize}
    %
    \item Dictionaries are super discrete structures with benefits!
      %
      \vspace*{-.2in}
      %
    \item Combine dictionaries and functions or use separately
      %
      \vspace*{-.2in}
      %
    \item Using dictionaries and functions in Python programs:
      %
      \begin{itemize}
        %
        \item {\bf Q1}: What are the characteristics of functions in Python?
          %
        \item {\bf Q2}: What are the characteristics of dictionaries in Python?
          %
        \item {\bf Q3}: How do you decide between a dictionary or a function?
          %
        \item {\bf Q4}: What functions can process a multiset discrete
          structure?
          %
        \item {\bf Q5}: How to combine functions and dictionaries for
          efficiency?
          %
      \end{itemize}
      %
      \vspace*{-.2in}
      %
    \item See \url{https://realpython.com/python-dicts/} for more on how to
      create and use dictionaries in Python!
      %
  \end{itemize}
  %
\end{frame}

\end{document}
