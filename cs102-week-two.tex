\documentclass[14pt,aspectratio=169]{beamer}

\usepackage{pgfpages}
\usepackage{fancyvrb}
\usepackage{tikz}
\usepackage{pgfplots}

\usetheme{auriga}
\usecolortheme{auriga}

\setbeamercolor{background canvas}{bg=lightgray}

% define some colors for a consistent theme across slides
\definecolor{red}{RGB}{181, 23, 0}
\definecolor{blue}{RGB}{0, 118, 186}
\definecolor{gray}{RGB}{146, 146, 146}

\title{Discrete Structures: \\ Basics of Mathematics and Programming}

\author{{\bf Gregory M. Kapfhammer}}

\institute[shortinst]{{\bf Department of Computer Science, Allegheny College}}

\begin{document}

{
  \setbeamercolor{page number in head/foot}{fg=background canvas.bg}
  \begin{frame}
    \titlepage
  \end{frame}
}

%% Slide
%
\begin{frame}{Technical Question}
  %
  \begin{center}
    %
    {\large How do I connect mathematical terminology (e.g., {\em mapping},
      {\em function}, {\em number}, and {\em set}), to the implementation of
      Python programs that declare and call functions and declare and manipulate
    variables?}
    %
  \end{center}
  %
  \vspace{2ex}
  %
  \begin{center}
    %
    \small Let's learn more about how the use of precise mathematical terms and
    concepts helps to effectively communicate and perform programming tasks!
    %
  \end{center}
  %
\end{frame}

% Slide
%
\begin{frame}{Discrete Structures = Math + Programming}
%
  \begin{itemize}
    %
    \item Discrete mathematics: symbols, character strings, truth values,
      objects, and collections of these entities
      %
    \item Specifying and designing a computer program
      \begin{itemize}
        \item Describe input, output, and internal objects
        \item Use the vocabulary of discrete mathematics
        \item Implement and test the program in a language
      \end{itemize}
      %
    \item Our goal: implement a program $P$ meets a specification $S$
      %
  \end{itemize}
%
\end{frame}

% Slide
%
\begin{frame}{Specifying a Program that Analyzes Web Pages}
  %
  \begin{itemize}
    %
    \item Informal specification: Read two web pages and then find and output
      all of the URLs that appear in them both
      %
    \item Different approaches to implementing this program
      %
      \begin{itemize}
        %
        \item Informal and intuitive specification
          %
        \item Precise mathematical specification
          %
        \item Which one is shorter? ... clearer? ... unambiguous?
      \end{itemize}
      %
    \item Mathematics helps us to implement a program that is correct,
      efficient, documented, and maintainable!
      %
  \end{itemize}
  %
\end{frame}

% Slide
%
\begin{frame}{Programs, Data, and Mathematical Objects}
  %
  \begin{itemize}
    %
    \item Our goal: Jump to different levels of abstraction (e.g., high-level
      versus low-level) when we create programs
      %
      \vspace*{-.1in}
      %
    \item What {\em is} a computer program?
      %
      \begin{itemize}
        %
        \item Informal or intuitive specification
          %
        \item Precise discrete mathematical specification
          %
        \item Realization of a specification in Python program
          %
        \item Bits packaged into bytes and words stored on a disk
          %
        \item A process in execution on a CPU and stored in memory
          %
      \end{itemize}
      %
      \vspace*{-.1in}
      %
    \item It is ``natural'' (and fun!) to jump from a discrete mathematical
      specification to a Python program
      %
  \end{itemize}
  %
\end{frame}

% Slide
%
\begin{frame}{Using Python to Explore Discrete Structures}
  %
  \begin{itemize}
    %
    \item Discrete structures support precise programming
      %
    \item Benefits of using Python to explore discrete structures
      %
      \begin{itemize}
        %
        \item Modern language with exceptional package support
          %
        \item Clean syntax and semantics that is easy to learn
          %
        \item Out-of-the-box support for many discrete structures
          %
      \end{itemize}
      %
    \item Download Python and start programming by visiting
      \url{https://www.python.org/}
      %
  \end{itemize}
  %
\end{frame}

\end{document}
