\documentclass[14pt,aspectratio=169]{beamer}

\usepackage{pgfpages}
\usepackage{fancyvrb}
\usepackage{tikz}
\usepackage{pgfplots}

\usetheme{auriga}
\usecolortheme{auriga}

\setbeamercolor{background canvas}{bg=lightgray}

% define some colors for a consistent theme across slides
\definecolor{red}{RGB}{181, 23, 0}
\definecolor{blue}{RGB}{0, 118, 186}
\definecolor{gray}{RGB}{146, 146, 146}

\title{Discrete Structures: \\ Basics of Mathematics and Programming}

\author{{\bf Gregory M. Kapfhammer}}

\institute[shortinst]{{\bf Department of Computer Science, Allegheny College}}

\begin{document}

{
  \setbeamercolor{page number in head/foot}{fg=background canvas.bg}
  \begin{frame}
    \titlepage
  \end{frame}
}

%% Slide
%
\begin{frame}{Technical Question}
  %
  \begin{center}
    %
    {\large How do I connect mathematical terminology (e.g., {\em mapping},
      {\em function}, {\em number}, and {\em set}), to the implementation of
      Python programs that declare and call functions and declare and manipulate
    variables?}
    %
  \end{center}
  %
  \vspace{2ex}
  %
  \begin{center}
    %
    \small Let's learn more about how the use of precise mathematical terms and
    concepts helps to effectively communicate and perform programming tasks!
    %
  \end{center}
  %
\end{frame}

% Slide
%
\begin{frame}{Version Control with Git and GitHub}
%
  \begin{itemize}
    %
    \item Benefits of version control systems
      %
    \item Using Git and GitHub
      \begin{itemize}
        \item Git tracks versions of files in directories
        \item GitHub enables sharing and collaboration
        \item Industry standard tools used in all assignments
      \end{itemize}
      %
    \item Challenges of version control systems
      %
  \end{itemize}
%
\end{frame}

% Slide
%
\begin{frame}{Editing Markdown and Python with VSCode}
%
  \begin{itemize}
    %
    \item Popularity of VSCode according to StackOverflow
      %
    \item Using the VSCode development environment
      \begin{itemize}
        \item Edit Markdown and Python files
        \item Commit changes to a GitHub repository
        \item Remote collaboration with the Live Share extension
      \end{itemize}
      %
    \item Get started by downloading VSCode from \url{https://code.visualstudio.com/}
      %
  \end{itemize}
%
\end{frame}

% Slide
%
\begin{frame}{Running GatorGrader and Python with Docker}
%
  \begin{itemize}
    %
    \item Popularity of Docker according to StackOverflow
      %
    \item Using the Docker during Python programming
      \begin{itemize}
        \item Download a Docker container from DockerHub
        \item Use a Docker container once to grade the work repository
        \item Enter into a Docker container to perform multiple commands
      \end{itemize}
      %
    \item Get started by downloading Docker from \url{https://docs.docker.com/desktop/}
      %
  \end{itemize}
%
\end{frame}

% Slide
%
\begin{frame}{Using Markdown and GitHub for Documentation}
%
  \begin{itemize}
    %
    \item A language for describing how to format rendered text
      %
    \item Benefits of using Markdown for documentation
      \begin{itemize}
        \item Simple and powerful approach to technical writing
        \item Formats both standard textual elements and source code
        \item GitHub natively renders Markdown in a Git repository
      \end{itemize}
      %
    \item Learn more about how to use Markdown by reading \url{https://www.markdownguide.org/}
      %
  \end{itemize}
%
\end{frame}

% Slide
%
\begin{frame}{Using Python to Explore Discrete Structures}
%
  \begin{itemize}
    %
    \item Discrete structures support precise programming
      %
    \item Benefits of using Python to explore discrete structures
      \begin{itemize}
        \item Modern language with exceptional package support
        \item Clean syntax and semantics that is easy to learn
        \item Out-of-the-box support for many discrete structures
      \end{itemize}
      %
    \item Download Python and start programming by visiting \url{https://www.python.org/}
      %
  \end{itemize}
%
\end{frame}

%% Slide
%
\begin{frame}{Technical Question}
  %
  \begin{center}
    %
    {\large How do I install and use the industry-standard programming tools that will
      help me to rigorously explore discrete structures with the Python
    programming language?}
    %
  \end{center}
  %
  \vspace{2ex}
  %
  \begin{center}
    %
    \small Let's look as some specific examples of how to use these tools!
    %
  \end{center}
  %
\end{frame}

% Slide
%
\begin{frame}{Creating a Clone of a GitHub Repository}
  %
  \setlength{\leftmarginii}{0.5cm}
  %
  \begin{itemize}
    %
    \item ``{\tt git clone}'' transfers from GitHub to your computer
      %
    \item Cloning a Git repository on GitHub
      %
      {\tiny
        \begin{itemize}
          \item {\tt git clone git@github.com:Allegheny-Computer-Science-102-F2020/cs102-F2020-plans.git}
            %
          \item {\tt git clone https://github.com/Allegheny-Computer-Science-102-F2020/cs102-F2020-plans.git}
        \end{itemize}
      }
      %
    \item What are the differences in these ``{\tt git clone}'' commands? What are the trade-offs in using them?
      %
  \end{itemize}
  %
\end{frame}

% Slide
%
\begin{frame}{Handy Commands to Support GitHub Use}
  %
  \begin{itemize}
    %
    \item ``{\tt git push -u origin master}'' transfers content from your computer
      to the GitHub servers
      %
    \item ``{\tt git commit cs102-week-one.tex}'' commits the changes in the
      file to your GitHub repository
      %
    \item ``{\tt git checkout -b feat/add-week-one-slides}'' creates a branch in
      a GitHub repository
      %
  \end{itemize}
  %
\end{frame}

\end{document}
