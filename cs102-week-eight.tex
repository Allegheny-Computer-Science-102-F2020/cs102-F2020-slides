\documentclass[14pt,aspectratio=169]{beamer}

\usepackage{pgfpages}
\usepackage{fancyvrb}
\usepackage{pgfplots}

\usepackage{minted}
\usemintedstyle{tango}

\usepackage{amsfonts}

\usepackage{moresize}
\usepackage{anyfontsize}

\usepackage{tikz}
\usetikzlibrary{arrows,shapes}
\usetikzlibrary{arrows.meta}

\tikzstyle{process}=[rectangle, draw, thick, text width=5em, text centered, minimum height=2.5em, fill=gray!40]
\tikzstyle{entity}=[rounded rectangle, draw, thick, text width=5em, text centered, minimum height=1.5em, fill=gray!40]

\usetheme{auriga}
\usecolortheme{auriga}

\setbeamercolor{background canvas}{bg=lightgray}

% define some colors for a consistent theme across slides
\definecolor{red}{RGB}{181, 23, 0}
\definecolor{blue}{RGB}{0, 118, 186}
\definecolor{gray}{RGB}{146, 146, 146}

\title{Discrete Structures: \\ Programming with \\ Streams in Python}

\author{{\bf Gregory M. Kapfhammer}}

\institute[shortinst]{{\bf Department of Computer Science, Allegheny College}}

\begin{document}

{
  \setbeamercolor{page number in head/foot}{fg=background canvas.bg}
  \begin{frame}
    \titlepage
  \end{frame}
}

%% Slide
%
\begin{frame}{Technical Question}
  %
  \hspace*{.25in}
  %
  \begin{minipage}{4.8in}
    %
    \vspace*{.1in}
    %
    \begin{center}
      %
      {\large How do I use dynamically generated streams of data to implement
      memory efficient and predictable Python programs?}
      %
    \end{center}
    %
  \end{minipage}
  %
  \vspace{2ex}
  %
  \begin{center}
    %
    \small Let's explore the difference between static sequences and dynamically
    generated streams! Let's better understand how the choice of a discrete
    structure influences the efficiency and behavior of Python programs!
    %
  \end{center}
  %
\end{frame}

% slide
%
\begin{frame}{Dynamically Generated Streams}
  %
  \begin{itemize}
    %
    \item Sequences and streams are evidentevident  in Python programs!
      %
      \vspace*{-.15in}
      %
    \item Static sequences and dynamically generated streams:
      %
      \begin{itemize}
        %
        \item A list is a static sequence that exists as a complete data
          structure
        %
        \item An file input stream appears dynamically over time when read
        %
        \item Let's distinguish between static sequences and dynamic streams
        %
        \item Example: streams are generated by iterators and range objects
        %
      \end{itemize}
      %
      \vspace*{-.2in}
      %
    \item What are the benefits of having a dynamic stream whose values are not
      known until then are generated and returned? How does the memory
      consumption of a Python program influence our choice of a discrete
      structure?
      %
  \end{itemize}
  %
\end{frame}

% Slide
%
\begin{frame}[fragile]
  \frametitle{File Input Involves the Use of Streams}
  \normalsize
  \hspace*{-.65in}
  \begin{minipage}{6in}
    \vspace*{.25in}
    \begin{minted}[mathescape, numbersep=5pt, fontsize=\large]{python}
    file = open("emails")
    for line in file:
      name, email = line.split(",")
      if name == "John Davis":
        print(email)
    \end{minted}
  \end{minipage}
  \vspace*{.25in}
  \begin{center}
    %
    \normalsize \noindent The file is a sequence of characters \\
    \normalsize \noindent A character is a sequence of numbers \\
    \normalsize \noindent Does the entire file exist in the computer's
    memory? \\
    %
  \end{center}
  %
\end{frame}

% Slide
%
\begin{frame}[fragile]
  \frametitle{Comprehensions and Generators are Different}
  \normalsize
  \begin{minipage}{6in}
    \vspace*{.2in}
    \begin{minted}[mathescape, numbersep=5pt, fontsize=\large]{python}
list_comprehension =
  ['Hello' for i in range(3)]
generator_expression =
  ('Hello' for i in range(3))

print(list_comprehension)
print(generator_expression)
    \end{minted}
  \end{minipage}
  \vspace*{.05in}
  \begin{center}
    %
    \normalsize \noindent What is the output of this Python program segment? \\
    %
  \end{center}
  %
\end{frame}

% Slide
%
\begin{frame}[fragile]
  \frametitle{Output from Comprehensions and Generators}
  \normalsize
  \begin{minipage}{6in}
    \vspace*{.2in}
    \begin{minted}[mathescape, numbersep=5pt, fontsize=\large]{text}

Output from the list comprehension:

  ['Hello', 'Hello', 'Hello']

Output from the generator function:

  <generator object <genexpr>
    at 0x7f0ebc7bc890>

    \end{minted}
    %
  \end{minipage}
  %
\end{frame}

% Slide
%
\begin{frame}[fragile]
  \frametitle{Using Generator Expressions in Python}
  \normalsize
  \begin{minipage}{6in}
    \vspace*{.2in}
    \begin{minted}[mathescape, numbersep=5pt, fontsize=\large]{python}
even_squares =
  (x * x for x in range(10)
             if x % 2 == 0)

print(even_squares)

for value in even_squares:
    print(value)
    \end{minted}
    %
  \end{minipage}
  %
\end{frame}

% Slide
%
\begin{frame}[fragile]
  \frametitle{Output from a Generator Expression in Python}
  \normalsize
  \hspace*{.1in}
  \begin{minipage}{6in}
    \vspace*{.2in}
    \begin{minted}[mathescape, numbersep=5pt, fontsize=\large]{text}
<generator object <genexpr>
  at 0x7f8cb00db430>

0
4
16
36
64
    \end{minted}
    %
  \end{minipage}
  %
\end{frame}

% Slide
%
\begin{frame}[fragile]
  \frametitle{Using Generator Functions in Python}
  \normalsize
  \begin{minipage}{6in}
    \vspace*{.1in}
    \begin{minted}[mathescape, numbersep=5pt, fontsize=\large]{python}
def fibonacci_generator(n):
    a = 1
    b = 1
    for i in range(n):
        yield a
        a, b = b, a + b

for value in fibonacci_generator(10):
    print(fibonacci_value, end=" ")
    \end{minted}
    %
  \end{minipage}
  %
\end{frame}

% Slide
%
\begin{frame}[fragile]
  \frametitle{Output from a Generator Function in Python}
  \normalsize
  \begin{minipage}{6in}
    \vspace*{.1in}
    \begin{minted}[mathescape, numbersep=5pt, fontsize=\large]{text}

<function fibonacci_generator
  at 0x7f791d4bb1f0>

1 1 2 3 5 8 13 21 34 55
    \end{minted}
    %
  \vspace*{.05in}
  \begin{center}
    %
    \normalsize \noindent What is this the output of the program segment? \\
    \normalsize \noindent Generator functions versus generator expressions? \\
    \normalsize \noindent Can you find the pattern in these numbers? \\
    \normalsize \noindent Benefits of generator expressions and functions? \\
    %
  \end{center}
  %
  \end{minipage}
  %
\end{frame}

% Slide
%
\begin{frame}[fragile]
  \frametitle{Using Functions and Tuples in Python}
  \normalsize
  \begin{minipage}{6in}
    \vspace*{.1in}
    \begin{minted}[mathescape, numbersep=5pt, fontsize=\large]{python}
def fibonacci_tuple(n):
    result = ( )
    a = 1
    b = 1
    for i in range(n):
        result += (a,)
        a, b = b, a + b
    return result
    \end{minted}
    %
  \end{minipage}
  %
\end{frame}

% Slide
%
\begin{frame}[fragile]
  \frametitle{Output from a Tuple Function in Python}
  \normalsize
  \begin{minipage}{6in}
    \vspace*{.1in}
    \begin{minted}[mathescape, numbersep=5pt, fontsize=\large]{text}
<function fibonacci_tuple
  at 0x7f6e976b61f0>

1 1 2 3 5 8 13 21 34 55

    \end{minted}
    %
  \vspace*{.05in}
  \begin{center}
    %
    \normalsize \noindent What is this the output of the program segment? \\
    \normalsize \noindent Generator functions versus tuple functions? \\
    \normalsize \noindent Notice that this function produces the same output! \\
    \normalsize \noindent Drawbacks of tuple creation functions? \\
    %
  \end{center}
  %
  \end{minipage}
  %
\end{frame}

% Slide
%
\begin{frame}{Comparing Sequences and n-Tuples in Python}
  %
  \begin{itemize}
    %
    \item Are sequences and n-tuples the same? Are they different?
      %
      \vspace*{-.15in}
      %
    \item Understanding the properties of n-tuples and sequences:
      %
      \begin{itemize}
        %
        \item Both n-tuples and sequences are ordered collections
        %
        \item Sequences are normally composed of the same type of data
        %
        \item n-tuples are not required to contain the same type of data
        %
        \item Sequences are not theoretically bounded in their size
        %
        \item n-tuples are theoretically bounded in their size
        %
        \item Both sequences and n-tuples are practically bounded in size
        %
      \end{itemize}
      %
      \vspace*{-.2in}
      %
    \item Do different types of sequences have common properties and behavior?
      Can we more generally understand them?
      %
  \end{itemize}
  %
\end{frame}

% Slide
%
\begin{frame}{Applying Monoids in Python Programs}
  %
  \begin{itemize}
    %
    \item Monoids are frequently used in Python programs
      %
      \vspace*{-.2in}
      %
    \item Python programs can use higher-order sequence functions
      %
      \vspace*{-.2in}
      %
    \item Using monoids and higher-order sequence functions:
      %
      \begin{itemize}
        %
        \item {\bf Q1}: What is the difference between a list and a tuple?
          %
        \item {\bf Q2}: How does a monoid generalize strings and integers?
          %
        \item {\bf Q3}: How do higher-order sequence functions use monoids?
          %
        \item {\bf Q4}: How can map-filter-reduce support parallel programming?
          %
        \item {\bf Q5}: What type of speedup will a parallel program achieve?
          %
      \end{itemize}
      %
      \vspace*{-.2in}
      %
    \item What are the ways in which the mathematical concept of a monoid
      connects to a wide variety of practical applications in the area of
      efficient and parallel computing?
      %
  \end{itemize}
  %
\end{frame}

\end{document}
