\documentclass[14pt,aspectratio=169]{beamer}

\usepackage{pgfpages}
\usepackage{fancyvrb}
\usepackage{pgfplots}

\usepackage{minted}
\usemintedstyle{tango}

\usepackage{amsfonts}

\usepackage{moresize}
\usepackage{anyfontsize}

\usepackage{tikz}
\usetikzlibrary{arrows,shapes}
\usetikzlibrary{arrows.meta}

\tikzstyle{process}=[rectangle, draw, thick, text width=5em, text centered, minimum height=2.5em, fill=gray!40]
\tikzstyle{entity}=[rounded rectangle, draw, thick, text width=5em, text centered, minimum height=1.5em, fill=gray!40]

\usetheme{auriga}
\usecolortheme{auriga}

\setbeamercolor{background canvas}{bg=lightgray}

% define some colors for a consistent theme across slides

\definecolor{red}{RGB}{181, 23, 0}
\definecolor{blue}{RGB}{0, 118, 186}
\definecolor{gray}{RGB}{146, 146, 146}

\title{Discrete Structures: \\ Programming with \\ Functions in Python}

\author{{\bf Gregory M. Kapfhammer}}

\institute[shortinst]{{\bf Department of Computer Science, Allegheny College}}

\begin{document}

{
  \setbeamercolor{page number in head/foot}{fg=background canvas.bg}
  \begin{frame}
    \titlepage
  \end{frame}
}

%% Slide
%
\begin{frame}{Technical Question}
  %
  \hspace*{.5in}
  %
  \begin{minipage}{4.3in}
    %
    \vspace*{.1in}
    %
    \begin{center}
      %
      {\large How do I use dictionaries, sets, and lists to efficiently
      implement mathematical functions in Python?}
      %
    \end{center}
    %
  \end{minipage}
  %
  \vspace{2ex}
  %
  \begin{center}
    %
    \small Let's learn how to pick the best discrete structure and function when
    we are implementing a function to calculate a mathematical calculation!
    Please review the content from previous weeks so that you can keep this
    week's content in an appropriate context. Also, try out all of the Python
    code!
    %
  \end{center}
  %
\end{frame}

% Slide
%
\begin{frame}{The Concept of a Mathematical Mapping}
  %
  \begin{itemize}
    %
    \item As first introduced in Chapter 1, a mapping is a set of ordered pairs
      in which no two have the same first element
      %
      \vspace*{-.15in}
      %
    \item Implementing functions in the Python language:
      %
      \begin{itemize}
        %
        \item Functions can accept zero or more inputs
        %
        \item Functions can produce zero or more outputs
        %
        \item Higher-order functions can accept and produce functions
        %
        \item One function's input can become another function's output
        %
        \item Imperative or declarative specification of functions
        %
        \item Discrete structures normally have a function interface
        %
      \end{itemize}
      %
      \vspace*{-.2in}
      %
    \item It is possible to implement a function in many different ways! How do
      we decide which approach is the best?
      %
  \end{itemize}
  %
\end{frame}

% Slide
%
\begin{frame}[fragile]
  \frametitle{Inputs and Outputs with The Collatz Function}
  \hspace*{-.2in}
  \begin{minipage}{6in}
    \vspace*{.25in}
    \begin{minted}[mathescape, numbersep=5pt, fontsize=\large]{python}
def collatz(number: int) -> Iterator[int]:
    yield number
    while number != 1:
        if number % 2 == 0:
            number = number // 2
        else:
            number = 3 * number + 1
        yield number
    \end{minted}
  \end{minipage}
  \vspace*{.25in}
  \begin{center}
    %
    \normalsize \noindent What is the output of this program? \\
    \normalsize \noindent Are there other ways to create a {\tt FiniteSet}? \\
    \normalsize \noindent What operations can we perform with a {\tt FiniteSet}? \\
    %
  \end{center}
  %
\end{frame}

% Slide
%
\begin{frame}[fragile]
  \frametitle{Inputs and Outputs with The Fibonacci Function}
  \hspace*{-.2in}
  \begin{minipage}{6in}
    \vspace*{.15in}
    \begin{minted}[mathescape, numbersep=5pt, fontsize=\large]{python}
def fibonacci(number: int) -> Tuple[int]:
    result = ()
    a = 1
    b = 1
    for i in range(number):
        result += (a,)
        a, b = b, a + b
    return result
    \end{minted}
  \end{minipage}
  \vspace*{.025in}
  \begin{center}
    %
    \normalsize \noindent What are the inputs and outputs of this function? \\
    %
  \end{center}
  %
\end{frame}

% Slide
%
\begin{frame}[fragile]
  \frametitle{Inputs and Outputs with The Factorial Function}
  \hspace*{-.2in}
  \begin{minipage}{6in}
    \vspace*{.2in}
    \begin{minted}[mathescape, numbersep=5pt, fontsize=\large]{python}
def factorial(number):
    factorial_value = 1
    values = list(range(1, number + 1))
    for value in values:
        factorial_value = factorial_value * value
    return factorial_value
    \end{minted}
  \end{minipage}
  \vspace*{.025in}
  \begin{center}
    %
    \normalsize \noindent How would you write the function's type signature? \\
    \normalsize \noindent What are the inputs and outputs of this function? \\
    \normalsize \noindent Are there other ways to implement this function? \\
    %
  \end{center}
  %
\end{frame}

% Slide
%
\begin{frame}{Using Python's Dictionaries to Create Mappings}
  %
  \begin{itemize}
    %
    \item Static sequences like a tuple or a list create a mapping!
      %
      \vspace*{-.15in}
      %
    \item Python's dictionary is a discrete structure with a mapping:
      %
      \begin{itemize}
        %
        \item Create a mapping between keys and values
        %
        \item Store key-value pairs inside of the structure
        %
        \item You ``lookup'' the key to ``find'' the value
        %
        \item Dictionary keys must be an immutable object
        %
        \item The value of the key can be of any data type
        %
        \item Integers and strings are common types for the key
        %
        \item Often called hashtables or hashmaps in other languages!
        %
      \end{itemize}
      %
      \vspace*{-.2in}
      %
    \item How can you create and use a dictionary in Python?
      %
  \end{itemize}
  %
\end{frame}

% Slide
%
\begin{frame}[fragile]
  \frametitle{Using Dictionaries to Create a Mapping: Way \#1}
  \hspace*{.1in}
  \begin{minipage}{6in}
    \vspace*{.25in}
    \begin{minted}[mathescape, numbersep=5pt, fontsize=\large]{python}
mlb_team = {
    'Colorado' : 'Rockies',
    'Boston'   : 'Red Sox',
    'Minnesota': 'Twins',
    'Milwaukee': 'Brewers',
    'Seattle'  : 'Mariners'
}
    \end{minted}
  \end{minipage}
  \vspace*{.025in}
  \begin{center}
    %
    \normalsize \noindent Reference: \url{https://realpython.com/python-dicts/} \\
    %
  \end{center}
  %
\end{frame}

% Slide
%
\begin{frame}[fragile]
  \frametitle{Using Dictionaries to Create a Mapping: Way \#2}
  \hspace*{.1in}
  \begin{minipage}{6in}
    \vspace*{.25in}
    \begin{minted}[mathescape, numbersep=5pt, fontsize=\large]{python}
mlb_team = dict([
    ('Colorado', 'Rockies'),
    ('Boston', 'Red Sox'),
    ('Minnesota', 'Twins'),
    ('Milwaukee', 'Brewers'),
    ('Seattle', 'Mariners')
])
    \end{minted}
  \end{minipage}
  \vspace*{.025in}
  \begin{center}
    %
    \normalsize \noindent Here is an alternative way to create key-value pairs! \\
    %
  \end{center}
  %
\end{frame}

% Slide
%
\begin{frame}[fragile]
  \frametitle{Using Dictionaries to Create a Mapping: Way \#3}
  \hspace*{.1in}
  \begin{minipage}{6in}
    \vspace*{.25in}
    \begin{minted}[mathescape, numbersep=5pt, fontsize=\large]{python}
MLB_team = dict(
    Colorado='Rockies',
    Boston='Red Sox',
    Minnesota='Twins',
    Milwaukee='Brewers',
    Seattle='Mariners'
)
    \end{minted}
  \end{minipage}
  \vspace*{.025in}
  \begin{center}
    %
    \normalsize \noindent How do I access a value given a specific key? \\
    %
  \end{center}
  %
\end{frame}

% Slide
%
\begin{frame}[fragile]
  \frametitle{Using Dictionaries in Python}
  \hspace*{.1in}
  \begin{minipage}{6in}
    \vspace*{.25in}
    \begin{minted}[mathescape, numbersep=5pt, fontsize=\large]{python}
print(type(MLB_team))
print(MLB_team)

print(MLB_team['Minnesota'])
print(MLB_team['Colorado'])

MLB_team['Kansas City'] = 'Royals'
    \end{minted}
  \end{minipage}
  \vspace*{.025in}
  \begin{center}
    %
    \normalsize \noindent What is the output of this program segment? \\
    %
  \end{center}
  %
\end{frame}

% Slide
%
\begin{frame}[fragile]
  \frametitle{Output Excerpt from Using a Dictionary}
  \begin{minipage}{6in}
    \vspace*{.25in}
    \begin{minted}[mathescape, numbersep=5pt, fontsize=\normalsize]{text}
<class 'dict'>
{'Colorado': 'Rockies', 'Boston': 'Red Sox',
 'Minnesota': 'Twins', 'Milwaukee': 'Brewers',
 'Seattle': 'Mariners'}

<class 'dict'>
{'Colorado': 'Rockies', 'Boston': 'Red Sox',
 'Minnesota': 'Twins', 'Milwaukee': 'Brewers',
 'Seattle': 'Mariners', 'Kansas City': 'Royals'}
    \end{minted}
  \end{minipage}
  \vspace*{.025in}
  \begin{center}
    %
    \normalsize \noindent What happens if you access data that does not exist? \\
    %
  \end{center}
  %
\end{frame}

% Slide
%
\begin{frame}[fragile]
  \frametitle{Asking a Dictionary for a Non-Existent Key}
  \begin{minipage}{6in}
    \vspace*{.25in}
    \begin{minted}[mathescape, numbersep=5pt, fontsize=\large]{python}

print(MLB_team['Toronto'])

Traceback (most recent call last):
  File "dictionary-example.py", line 32, in <module>
    print(MLB_team['Toronto'])
KeyError: 'Toronto'

    \end{minted}
  \end{minipage}
  \vspace*{.025in}
  \begin{center}
    %
    \normalsize \noindent Why did this program crash? What does the error mean?\\
    %
  \end{center}
  %
\end{frame}

% Slide
%
\begin{frame}{Should You Pick a Dictionary or a Function?}
  %
  \begin{itemize}
    %
    \item Wait, are dictionaries and functions the same? Well ...
      %
      \vspace*{-.15in}
      %
    \item Picking between a discrete structure and a computation:
      %
      \begin{itemize}
        %
        \item A discrete structure must be finite and fit into memory
        %
        \item A function might be the only viable solution for many keys
        %
        \item A discrete structure can often be mutable, giving flexibility
        %
        \item It is harder to make a function mutable, in most languages
        %
        \item The lookup of a value for a key is done efficiently in a
          dictionary
        %
        \item A function that computes the value for a key may take time
        %
        \item But, what is going to be easier to program and test?
        %
      \end{itemize}
      %
      \vspace*{-.2in}
      %
    \item The honest answer is ``it depends''! Use judgement and benchmarks!
      Defend your decision with evidence!
      %
  \end{itemize}
  %
\end{frame}

% Slide
%
\begin{frame}{Creating Mathematical Mappings in Python}
  %
  \begin{itemize}
    %
    \item Sets are a discrete structure with many practical benefits!
      %
      \vspace*{-.2in}
      %
    \item Programming with sets lets you calculate probabilities
      %
      \vspace*{-.2in}
      %
    \item Using finite sets and probability in Python programs:
      %
      \begin{itemize}
        %
        \item {\bf Q1}: What are the characteristics of an infinite set? Finite set?
          %
        \item {\bf Q2}: What are the built-in operations provided by a finite set?
          %
        \item {\bf Q3}: How does the {\tt sympy} package support set
          programming?
          %
        \item {\bf Q4}: How does the use of sets support probability
          calculations?
          %
        \item {\bf Q5}: How can Venn diagrams help to visualize set
          relationships?
          %
      \end{itemize}
      %
      \vspace*{-.2in}
      %
    \item See \url{https://realpython.com/python-dicts/} for more on how to
      create and use dictionaries in Python!
      %
  \end{itemize}
  %
\end{frame}

\end{document}
