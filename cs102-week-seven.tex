\documentclass[14pt,aspectratio=169]{beamer}

\usepackage{pgfpages}
\usepackage{fancyvrb}
\usepackage{pgfplots}

\usepackage{minted}
\usemintedstyle{tango}

\usepackage{amsfonts}

\usepackage{moresize}
\usepackage{anyfontsize}

\usepackage{tikz}
\usetikzlibrary{arrows,shapes}
\usetikzlibrary{arrows.meta}

\tikzstyle{process}=[rectangle, draw, thick, text width=5em, text centered, minimum height=2.5em, fill=gray!40]
\tikzstyle{entity}=[rounded rectangle, draw, thick, text width=5em, text centered, minimum height=1.5em, fill=gray!40]

\usetheme{auriga}
\usecolortheme{auriga}

\setbeamercolor{background canvas}{bg=lightgray}

% define some colors for a consistent theme across slides
\definecolor{red}{RGB}{181, 23, 0}
\definecolor{blue}{RGB}{0, 118, 186}
\definecolor{gray}{RGB}{146, 146, 146}

\title{Discrete Structures: \\ Applications of Sequences, Monoids, and
Lists in Python}

\author{{\bf Gregory M. Kapfhammer}}

\institute[shortinst]{{\bf Department of Computer Science, Allegheny College}}

\begin{document}

{
  \setbeamercolor{page number in head/foot}{fg=background canvas.bg}
  \begin{frame}
    \titlepage
  \end{frame}
}

%% Slide
%
\begin{frame}{Technical Question}
  %
  \hspace*{.25in}
  %
  \begin{minipage}{4.8in}
    %
    \vspace*{.1in}
    %
    \begin{center}
      %
      {\large How do I employ the mathematical concepts of sequences,
        monoids, and lists to implement efficient Python programs that use
        functions with a clearly specified behavior to perform tasks like
        finding a name in a file or computing the arithmetic mean of data values?}
      %
    \end{center}
    %
  \end{minipage}
  %
  \vspace{2ex}
  %
  \begin{center}
    %
    \small Let's learn how to translate concepts from discrete mathematics to
    implement efficient Python programs that are rigorously specified!
    %
  \end{center}
  %
\end{frame}

% Slide
%
\begin{frame}{Reviewing Examples of Sequences in Python}
  %
  \begin{itemize}
    %
    \item Sequences are commonly found in Python programs!
      %
      \vspace*{-.15in}
      %
    \item Examples of the sequence discrete structure in Python:
      %
      \begin{itemize}
        %
        \item A string is a sequence of individual characters
        %
        \item The {\tt range(20)} function returns a sequence of numbers
        %
        \item Files are sequences of lines containing content
        %
        \item Each line in a file is a sequence of individual characters
        %
        \item Each individual character is a sequence of numbers
        %
        \item Each individual number is a sequence of binary digits
        %
      \end{itemize}
      %
      \vspace*{-.2in}
      %
    \item Do these sequences all have properties in common? If they do, then
      that would help us to generally understand them!
      %
  \end{itemize}
  %
\end{frame}

% Slide
%
\begin{frame}{Comparing Sequences and n-Tuples in Python}
  %
  \begin{itemize}
    %
    \item Are sequences and n-tuples the same? Are they different?
      %
      \vspace*{-.15in}
      %
    \item Understanding the properties of n-tuples and sequences:
      %
      \begin{itemize}
        %
        \item Both n-tuples and sequences are ordered collections
        %
        \item Sequences are normally composed of the same type of data
        %
        \item n-tuples are not required to contain the same type of data
        %
        \item Sequences are not theoretically bounded in their size
        %
        \item n-tuples are theoretically bounded in their size
        %
        \item Both sequences and n-tuples are practically bounded in size
        %
      \end{itemize}
      %
      \vspace*{-.2in}
      %
    \item Do different types of sequences have common properties and behavior?
      Can we more generally understand them?
      %
  \end{itemize}
  %
\end{frame}

% Slide
%
\begin{frame}[fragile]
  \frametitle{String Concatenation in Python}
  \begin{minipage}{6in}
    \vspace*{.25in}
    \begin{minted}[mathescape, numbersep=5pt, fontsize=\large]{python}
hello = "hello"
world = "world"
space = " "
message = hello + space + world
print(f"The message is: {message}")
    \end{minted}
  \end{minipage}
  \vspace*{.05in}
  \begin{center}
    %
    \normalsize \noindent You can concatenate or ``glue together'' strings \\
    %
    \normalsize \noindent What would happen if you picked a different order?\\
    %
  \end{center}
  %
\end{frame}

% Slide
%
\begin{frame}[fragile]
  \frametitle{Reversed String Concatenation in Python}
  \begin{minipage}{6in}
    \vspace*{.25in}
    \begin{minted}[mathescape, numbersep=5pt, fontsize=\large]{python}
hello = "hello"
world = "world"
space = " "
message = world + space + hello
print(f"The message is: {message}")
    \end{minted}
  \end{minipage}
  \vspace*{.05in}
  \begin{center}
    %
    \normalsize \noindent What is the output of this program segment? \\
    %
    \normalsize \noindent How ``f-strings'' make it easy to produce output? \\
    %
  \end{center}
  %
\end{frame}

% Slide
%
\begin{frame}{Applying n-Tuples and Lists in Python Programs}
  %
  \begin{itemize}
    %
    \item n-tuples and lists are frequently used in Python programs
      %
      \vspace*{-.2in}
      %
    \item Python programs can use CSV files and databases
      %
      \vspace*{-.2in}
      %
    \item Using n-tuples inside of a Python program:
      %
      \begin{itemize}
        %
        \item {\bf Q1}: What is the difference between a list and a tuple?
          %
        \item {\bf Q2}: How do you read a CSV file from the file system?
          %
        \item {\bf Q3}: How do you parse a CSV file encoded in a string?
          %
        \item {\bf Q4}: How do you handle the intricacies of real-world CSV
          files?
          %
        \item {\bf Q5}: How can a Python program connect to a SQLite database?
          %
      \end{itemize}
      %
      \vspace*{-.2in}
      %
    \item What are the ways in which the  mathematical concept of an n-tuple
      connects to a wide variety of practical applications in computing,
      business, and science?
      %
  \end{itemize}
  %
\end{frame}

\end{document}
