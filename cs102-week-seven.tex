\documentclass[14pt,aspectratio=169]{beamer}

\usepackage{pgfpages}
\usepackage{fancyvrb}
\usepackage{pgfplots}

\usepackage{minted}
\usemintedstyle{tango}

\usepackage{amsfonts}

\usepackage{moresize}
\usepackage{anyfontsize}

\usepackage{tikz}
\usetikzlibrary{arrows,shapes}
\usetikzlibrary{arrows.meta}

\tikzstyle{process}=[rectangle, draw, thick, text width=5em, text centered, minimum height=2.5em, fill=gray!40]
\tikzstyle{entity}=[rounded rectangle, draw, thick, text width=5em, text centered, minimum height=1.5em, fill=gray!40]

\usetheme{auriga}
\usecolortheme{auriga}

\setbeamercolor{background canvas}{bg=lightgray}

% define some colors for a consistent theme across slides
\definecolor{red}{RGB}{181, 23, 0}
\definecolor{blue}{RGB}{0, 118, 186}
\definecolor{gray}{RGB}{146, 146, 146}

\title{Discrete Structures: \\ Applications of Sequences, Monoids, and
Lists in Python}

\author{{\bf Gregory M. Kapfhammer}}

\institute[shortinst]{{\bf Department of Computer Science, Allegheny College}}

\begin{document}

{
  \setbeamercolor{page number in head/foot}{fg=background canvas.bg}
  \begin{frame}
    \titlepage
  \end{frame}
}

%% Slide
%
\begin{frame}{Technical Question}
  %
  \hspace*{.25in}
  %
  \begin{minipage}{4.8in}
    %
    \vspace*{.1in}
    %
    \begin{center}
      %
      {\large How do I employ the mathematical concepts of sequences,
        monoids, and lists to implement efficient Python programs that use
        functions with a clearly specified behavior to perform tasks like
        finding a name in a file or computing the arithmetic mean of data values?}
      %
    \end{center}
    %
  \end{minipage}
  %
  \vspace{2ex}
  %
  \begin{center}
    %
    \small Let's learn how to translate concepts from discrete mathematics to
    implement efficient Python programs that are rigorously specified!
    %
  \end{center}
  %
\end{frame}

% Slide
%
\begin{frame}{Reviewing Examples of Sequences in Python}
  %
  \begin{itemize}
    %
    \item Sequences are commonly found in Python programs!
      %
      \vspace*{-.15in}
      %
    \item Examples of the sequence discrete structure in Python:
      %
      \begin{itemize}
        %
        \item A string is a sequence of individual characters
        %
        \item The {\tt range(20)} function returns a sequence of numbers
        %
        \item Files are sequences of lines containing content
        %
        \item Each line in a file is a sequence of individual characters
        %
        \item Each individual character is a sequence of numbers
        %
        \item Each individual number is a sequence of binary digits
        %
      \end{itemize}
      %
      \vspace*{-.2in}
      %
    \item Do these sequences all have properties in common? If they do, then
      that would help us to generally understand them!
      %
  \end{itemize}
  %
\end{frame}

% Slide
%
\begin{frame}{Comparing Sequences and n-Tuples in Python}
  %
  \begin{itemize}
    %
    \item Are sequences and n-tuples the same? Are they different?
      %
      \vspace*{-.15in}
      %
    \item Understanding the properties of n-tuples and sequences:
      %
      \begin{itemize}
        %
        \item Both n-tuples and sequences are ordered collections
        %
        \item Sequences are normally composed of the same type of data
        %
        \item n-tuples are not required to contain the same type of data
        %
        \item Sequences are not theoretically bounded in their size
        %
        \item n-tuples are theoretically bounded in their size
        %
        \item Both sequences and n-tuples are practically bounded in size
        %
      \end{itemize}
      %
      \vspace*{-.2in}
      %
    \item Do different types of sequences have common properties and behavior?
      Can we more generally understand them?
      %
  \end{itemize}
  %
\end{frame}

% Slide
%
\begin{frame}[fragile]
  \frametitle{String Concatenation in Python}
  \begin{minipage}{6in}
    \vspace*{.25in}
    \begin{minted}[mathescape, numbersep=5pt, fontsize=\large]{python}
hello = "hello"
world = "world"
space = " "
message = hello + space + world
print(f"The message is: {message}")
    \end{minted}
  \end{minipage}
  \vspace*{.05in}
  \begin{center}
    %
    \normalsize \noindent You can concatenate or ``glue together'' strings \\
    %
    \normalsize \noindent What would happen if you picked a different order?\\
    %
  \end{center}
  %
\end{frame}

% Slide
%
\begin{frame}[fragile]
  \frametitle{Reversed String Concatenation in Python}
  \begin{minipage}{6in}
    \vspace*{.25in}
    \begin{minted}[mathescape, numbersep=5pt, fontsize=\large]{python}
hello = "hello"
world = "world"
space = " "
message = world + space + hello
print(f"The message is: {message}")
    \end{minted}
  \end{minipage}
  \vspace*{.05in}
  \begin{center}
    %
    \normalsize \noindent What is the output of this program segment? \\
    %
    \normalsize \noindent How ``f-strings'' make it easy to produce output? \\
    %
  \end{center}
  %
\end{frame}

% Slide
%
\begin{frame}[fragile]
  \frametitle{Empty String Concatenation in Python}
  \begin{minipage}{6in}
    \vspace*{.25in}
    \begin{minted}[mathescape, numbersep=5pt, fontsize=\large]{python}
world = "world"
empty = ""
message = empty + world
print(f"The message is: {message}")
    \end{minted}
  \end{minipage}
  \vspace*{.05in}
  \begin{center}
    %
    \normalsize \noindent The {\tt empty} variable is an identity string \\
    %
    \normalsize \noindent What is the output of this program segment? \\
    %
    \normalsize \noindent What if we switched the order of the concatenation? \\
    %
  \end{center}
  %
\end{frame}

% Slide
%
\begin{frame}[fragile]
  \frametitle{Reversed Empty String Concatenation in Python}
  \begin{minipage}{6in}
    \vspace*{.25in}
    \begin{minted}[mathescape, numbersep=5pt, fontsize=\large]{python}
world = "world"
empty = ""
message = world + empty
print(f"The message is: {message}")
    \end{minted}
  \end{minipage}
  \vspace*{.05in}
  \begin{center}
    %
    \normalsize \noindent What is the output of this program segment? \\
    %
    \normalsize \noindent Can we generalize these observations about strings? \\
    %
    \normalsize \noindent Define new discrete structure with predictable properties? \\
    %
  \end{center}
  %
\end{frame}

% Slide
%
\begin{frame}{Characterizing String Concatenation in Python}
  %
  \begin{itemize}
    %
    \item Define $S$ to be the set of all possible strings
      %
      \vspace*{-.15in}
      %
    \item What properties of $S$ are always true?
      %
      \begin{itemize}
        %
        \item For $s_1, s_2 \in S$ and the concatenation operator ``$+$'', $s_1
          + s_2 \in S$
          %
        \item For $s_1, s_2 \in S$ and the concatenation operator ``$+$'', $s_2
          + s_1 \in S$
          %
        \item For $s_1, s_2, s_3 \in S$, ``$+$'' is associative: $(s_1 + s_2) +
          s_3 = s_1 + (s_2 + s_3)$
          %
        \item For $s_1, s_2, \in S$, ``$+$'' is not commutative: $(s_1 + s_2)
          \neq s_2 + s_1$
          %
        \item For $s_1, s_2, \in S$, if $s_1 = s_2$ or either $s_1 = \epsilon$,
          then ``$+$'' is commutative
          %
      \end{itemize}
      %
      \vspace*{-.2in}
      %
    \item These properties of strings help us to generalize and understand
      their behavior! Let's explore further!
      %
  \end{itemize}
  %
\end{frame}

% Slide
%
\begin{frame}{Motivating the Monoid Discrete Structure}
  %
  \begin{itemize}
    %
    \item Wow, strings and integers have similar properties!
      %
      \vspace*{-.15in}
      %
    \item The integers and the mathematical operation of ``$+$'':
      %
      \begin{itemize}
        %
        \item Adding two integers results in another integer
          %
        \item Addition of integers adheres to the associative property
          %
        \item Addition has an identity of $0$ because $0 + n = n + 0$
          %
        \item Key question: are integers and strings of the same general type?
          %
      \end{itemize}
      %
      \vspace*{-.2in}
      %
    \item {\bf Monoid}: A set that has an associative binary operator and an
      identity element. Can we explain this more formally?
      %
      \vspace*{-.2in}
      %
    \item If we know that a discrete structure used in a Python program is a
      monoid then we can understand its behavior!
      %
  \end{itemize}
  %
\end{frame}

% Slide
%
\begin{frame}{Understanding the Monoid Discrete Structure}
  %
  \begin{itemize}
    %
    \item A monoid is an ordered pair $(S, \otimes)$ for a set $S$ and any
      binary operator $\otimes$ that satisfies the following conditions:
      %
      \begin{itemize}
        %
        \item {\bf Type Preservation}: $\forall s_1, s_2 \in S$, $s_1 \otimes
          s_2 \in S$
          %
        \item {\bf Associative Property}: $\forall s_1, s_2, s_3 \in S$, $(s_1
          \otimes s_2) \otimes s_3 = s_1 \otimes (s_2 \otimes s_3)$
          %
        \item {\bf Identity Element}: $\exists \epsilon \in S$, such that
          $\forall s \in S, \epsilon \otimes s = s$ and $s \otimes \epsilon = s$
          %
      \end{itemize}
      %
      \vspace*{-.2in}
      %
    \item We often say that $S$ is a monoid under $\otimes$ with identity
      $\epsilon$
      %
      \vspace*{-.2in}
      %
    \item If this is confusing, remember that the discrete structure called a
      monoid is a generalization of strings and integers!
      %
      \vspace*{-.2in}
      %
    \item If you know how strings behave then you understand the monoid;
      ``monoid'' describes ``string like'' structures
      %
  \end{itemize}
  %
\end{frame}

% Slide
%
\begin{frame}{Applying the Monoid Discrete Structure}
  %
  \begin{itemize}
    %
    \item Monoids frequently exist inside of our Python programs!
      %
      \vspace*{-.2in}
      %
    \item Monoids give a vocabulary to describe discrete structures
      %
      \vspace*{-.2in}
      %
    \item Saying that a structure ``is a monoid under a specific identity''
      gives us a precise understanding of its behavior
      %
      \vspace*{-.2in}
      %
    \item Knowing that a discrete structure is a monoid means that we can assume
      that associativity holds for its operator
      %
      \vspace*{-.2in}
      %
    \item That is why, for integers, we can have ``big operators'' like $\sum$
      for adding numbers and $\prod$ for multiplying numbers
      %
  \end{itemize}
  %
\end{frame}

% Slide
%
\begin{frame}[fragile]
  \frametitle{Sums of Lists in Python}
  \begin{minipage}{6in}
    \vspace*{.25in}
    \begin{minted}[mathescape, numbersep=5pt, fontsize=\large]{python}
standard_list = [1, 2, 3, 4, 5]
reversed_list = [5, 4, 3, 2, 1]

sum_list = sum(standard_list)
sum_reversed_list = sum(reversed_list)
    \end{minted}
  \end{minipage}
  \vspace*{.05in}
  \begin{center}
    %
    \normalsize \noindent What is the output of this program segment? \\
    %
    \normalsize \noindent This code demonstrates the behavior of the $\sum$
    operator \\
    %
    \normalsize \noindent You can sum numbers in any order and get the same
    answer! \\
    %
  \end{center}
  %
\end{frame}

% Slide
%
\begin{frame}[fragile]
  \frametitle{Products of Lists in Python}
  \begin{minipage}{6in}
    \vspace*{.25in}
    \begin{minted}[mathescape, numbersep=5pt, fontsize=\large]{python}
import math
standard_list = [1, 2, 3, 4, 5]
reversed_list = [5, 4, 3, 2, 1]
product_list = math.prod(standard_list)
product_reversed_list =
    math.prod(reversed_list)
    \end{minted}
  \end{minipage}
  \vspace*{.05in}
  \begin{center}
    %
    \normalsize \noindent This code demonstrates the behavior of the $\prod$
    operator \\
    %
    \normalsize \noindent Multiplying numbers in any order and gets the same
    answer! \\
    %
  \end{center}
  %
\end{frame}

% Slide
%
\begin{frame}[fragile]
  \frametitle{Should I Use a List or a Tuple for Storage?}
  \begin{minipage}{6in}
    \vspace*{.15in}
    \begin{minted}[mathescape, numbersep=5pt, fontsize=\normalsize]{text}
1970-01-01,81.342
1971-01-01,83.300
1972-01-01,84.700
1973-01-01,85.500
1974-01-01,86.100
1975-01-01,87.000
1976-01-01,87.600
1977-01-01,87.600
1978-01-01,88.000
1979-01-01,88.100
    \end{minted}
  \end{minipage}
  \vspace*{.05in}
  \begin{center}
    %
    \normalsize \noindent CSV file stores ordered pairs of dates and
    population counts\\
    %
  \end{center}
  %
\end{frame}

% Slide
%
\begin{frame}[fragile]
  \frametitle{Using Mutable Lists in Python}
  \begin{minipage}{6in}
    \vspace*{.15in}
    \begin{minted}[mathescape, numbersep=5pt, fontsize=\large]{python}

  data_number_list = []
  for line in data_text.splitlines():
      ordered_pair = line.split(",")
      data_number_list.append(float(
              ordered_pair[1]))
  return data_number_list

    \end{minted}
  \end{minipage}
  %
  \vspace*{.025in}
    %
  \begin{center}
    %
    \normalsize \noindent Both lists and tuples and examples of sequences! \\
    %
    \normalsize \noindent What are the differences between lists and tuples? \\
    %
  \end{center}
  %
\end{frame}

% Slide
%
\begin{frame}{Higher-Order Sequence Functions}
  %
  \begin{itemize}
    %
    \item Define higher-order functions that work for any sequence
      %
      \vspace*{-.15in}
      %
    \item These functions should work for lists, ordered pairs, tuples
      %
      \begin{itemize}
        %
        \item {\tt map}: Apply a function to every element of a sequence
          %
        \item {\tt filter}: Apply a boolean function to every element of a
          sequence, returning only those matching the filter's rules
          %
        \item {\tt reduce}: Apply a function that acts like a binary operator to
          a sequence of values, combining them to a single value
          %
      \end{itemize}
      %
      \vspace*{-.2in}
      %
    \item These three operators give a vocabulary for implementing complex
      programs in a functional programming style
      %
      \vspace*{-.2in}
      %
    \item Let's explore each of these operators in greater detail!
      %
  \end{itemize}
  %
\end{frame}

% Slide
%
\begin{frame}[fragile]
  \frametitle{Using the Map Function with a Literal Tuple}
  \begin{minipage}{6in}
    \vspace*{.15in}
    \begin{minted}[mathescape, numbersep=5pt, fontsize=\normalsize]{python}
def square(value):
    return value * value

def map(f, sequence):
    result = (  )
    for element in sequence:
        result += ( f(element), )
    return result

squared = map(square, (2, 3, 5, 7, 11))
print(squared)
    \end{minted}
  \end{minipage}
  %
\end{frame}

% Slide
%
\begin{frame}[fragile]
  \frametitle{Using the Map Function with a Sequence}
  \begin{minipage}{6in}
    \vspace*{.15in}
    \begin{minted}[mathescape, numbersep=5pt, fontsize=\normalsize]{python}
def square(value):
    return value * value

def map(f, sequence):
    result = (  )
    for element in sequence:
        result += ( f(element), )
    return result

squared_range = map(square, range(10))
print(squared_range)
    \end{minted}
  \end{minipage}
  %
\end{frame}

% Slide
%
\begin{frame}[fragile]
  \frametitle{Filtering Even Numbers from a Tuple}
  \begin{minipage}{6in}
    \vspace*{.15in}
    \begin{minted}[mathescape, numbersep=5pt, fontsize=\large]{python}
def is_even(value):
    if value % 2 == 0:
        return True
    return False

filtered_even = filter(is_even,
    (2, 3, 5, 7, 11))
print(list(filtered_even))
    \end{minted}
  \end{minipage}
  %
\end{frame}

% Slide
%
\begin{frame}[fragile]
  \frametitle{Filtering Odd Numbers from a Tuple}
  \begin{minipage}{6in}
    \vspace*{.15in}
    \begin{minted}[mathescape, numbersep=5pt, fontsize=\large]{python}
def is_odd(value):
    if value % 2 == 1:
        return True
    return False

filtered_odd = filter(is_odd,
    (2, 3, 5, 7, 11))
print(list(filtered_odd))
    \end{minted}
  \end{minipage}
  %
\end{frame}

% Slide
%
\begin{frame}[fragile]
  \frametitle{Summations Through the Application of Reduce}
  \begin{minipage}{6in}
    \vspace*{.15in}
    \begin{minted}[mathescape, numbersep=5pt, fontsize=\normalsize]{python}
def plus(number_one, number_two):
    return number_one + number_two

def reduce(f, sequence, initial):
    result = initial
    for value in sequence:
        result = f(result, value)
    return result

numbers = [1, 2, 3, 4, 5]
added_numbers = reduce(plus, numbers, 0)
    \end{minted}
  \end{minipage}
  %
\end{frame}

% Slide
%
\begin{frame}{Connecting Monoids and Map-Reduce-Filter}
  %
  \begin{itemize}
    %
    \item These higher-order sequence functions are independent and free of
      ``side effects'' and thus can be parallelized
      %
      \vspace*{-.15in}
      %
    \item Since a monoid has the associativity property we can apply the {\tt
      map}, {\tt filter}, and {\tt reduce} operators in parallel and then
      combine the solution, often achieving a speedup
      %
      \vspace*{-.15in}
      %
    \item These three operators give a vocabulary for implementing complex
      programs in a functional programming style
      %
      \vspace*{-.15in}
      %
    \item Parallel computation is important given the diminishing
      returns associated with sequential computation
      %
  \end{itemize}
  %
\end{frame}

% Slide
%
\begin{frame}{Applying Monoids in Python Programs}
  %
  \begin{itemize}
    %
    \item Monoids are frequently used in Python programs
      %
      \vspace*{-.2in}
      %
    \item Python programs can use higher-order sequence functions
      %
      \vspace*{-.2in}
      %
    \item Using monoids and higher-order sequence functions:
      %
      \begin{itemize}
        %
        \item {\bf Q1}: What is the difference between a list and a tuple?
          %
        \item {\bf Q2}: How does a monoid generalize strings and integers?
          %
        \item {\bf Q3}: How do higher-order sequence functions use monoids?
          %
        \item {\bf Q4}: How can map-filter-reduce support parallel programming?
          %
        \item {\bf Q5}: What type of speedup will a parallel program achieve?
          %
      \end{itemize}
      %
      \vspace*{-.2in}
      %
    \item What are the ways in which the mathematical concept of a monoid
      connects to a wide variety of practical applications in the fields of
      efficient and parallel computing?
      %
  \end{itemize}
  %
\end{frame}

\end{document}
