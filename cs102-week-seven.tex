\documentclass[14pt,aspectratio=169]{beamer}

\usepackage{pgfpages}
\usepackage{fancyvrb}
\usepackage{pgfplots}

\usepackage{minted}
\usemintedstyle{tango}

\usepackage{amsfonts}

\usepackage{moresize}
\usepackage{anyfontsize}

\usepackage{tikz}
\usetikzlibrary{arrows,shapes}
\usetikzlibrary{arrows.meta}

\tikzstyle{process}=[rectangle, draw, thick, text width=5em, text centered, minimum height=2.5em, fill=gray!40]
\tikzstyle{entity}=[rounded rectangle, draw, thick, text width=5em, text centered, minimum height=1.5em, fill=gray!40]

\usetheme{auriga}
\usecolortheme{auriga}

\setbeamercolor{background canvas}{bg=lightgray}

% define some colors for a consistent theme across slides
\definecolor{red}{RGB}{181, 23, 0}
\definecolor{blue}{RGB}{0, 118, 186}
\definecolor{gray}{RGB}{146, 146, 146}

\title{Discrete Structures: \\ Applications of Sequences, Monoids, and
Lists in Python}

\author{{\bf Gregory M. Kapfhammer}}

\institute[shortinst]{{\bf Department of Computer Science, Allegheny College}}

\begin{document}

{
  \setbeamercolor{page number in head/foot}{fg=background canvas.bg}
  \begin{frame}
    \titlepage
  \end{frame}
}

%% Slide
%
\begin{frame}{Technical Question}
  %
  \hspace*{.25in}
  %
  \begin{minipage}{4.8in}
    %
    \vspace*{.1in}
    %
    \begin{center}
      %
      {\large How do I employ the mathematical concepts of sequences,
        monoids, and lists to implement efficient Python programs that use
        functions with a clearly specified behavior to perform tasks like
        finding a name in a file or computing the arithmetic mean of data values?}
      %
    \end{center}
    %
  \end{minipage}
  %
  \vspace{2ex}
  %
  \begin{center}
    %
    \small Let's learn how to translate concepts from discrete mathematics to
    implement efficient Python programs that are rigorously specified!
    %
  \end{center}
  %
\end{frame}

% Slide
%
\begin{frame}{Understanding Ordered Pairs}
  %
  \begin{itemize}
    %
    \item Mathematical concepts yield predictable programs
      %
      \vspace*{-.15in}
      %
    \item Understanding the concept of an ``ordered pair''
      %
      \begin{itemize}
        %
        \item {\bf Pair}: grouping of two entities
          %
        \item {\bf Ordered}: order of entities matters
          %
        \item {\bf Ordered Pair}: grouping of two entities for which order
          matters
          %
        \item {\bf Coordinate on Earth}: latitude and longitude are an ordered
          pair
          %
        \item {\bf Complex Numbers}: real and imaginary parts are an ordered
          pair
          %
        \item An ordered pair is not the same as a set of two elements!
          %
      \end{itemize}
      %
      \vspace*{-.2in}
      %
    \item If we have ordered pairs of entities, can we generalize to
      an ordered grouping beyond two entities? How?
      %
  \end{itemize}
  %
\end{frame}

% Slide
%
\begin{frame}[fragile]
  \frametitle{Practical Applications of Ordered Pairs}
  % \hspace*{-.15in}
  \begin{minipage}{6in}
    \vspace*{.25in}
    \begin{minted}[mathescape, numbersep=5pt, fontsize=\large]{text}
(40.7580° N, 73.9855° W)
  Times Square
(60.1699° N, 24.9384° E)
  Helsinki, Finland
(90.0° N, 135.0000° W)
  North Pole
    \end{minted}
  \end{minipage}
  \vspace*{.05in}
  \begin{center}
    %
    \normalsize \noindent Latitude and longitude provide a ``global address''
    for a location\\
    %
    \normalsize \noindent Why does the order matter for these pairs of location
    data?\\
    %
  \end{center}
  %
\end{frame}

% Slide
%
\begin{frame}{Generalizing Ordered Pairs to n-Tuples}
  %
  \begin{itemize}
    %
    \item We could have an ``ordered triple'' or ``ordered quadruple''
      %
      \vspace*{-.15in}
      %
    \item The n-tuple is the collective name for ``tuples'' of any size
      %
      \begin{itemize}
        %
        \item A 2-tuple is the same as an ordered pair
          %
        \item A 3-tuple is the same as an ordered triple
          %
        \item A 4-tuple is the same as an ordered quadruple
          %
      \end{itemize}
      %
      \vspace*{-.2in}
      %
    \item Write n-tuples with notation like $(1,2)$ or $(x,y,z)$
      %
      \vspace*{-.2in}
    \item n-tuples enable creation of new mathematical objects
      %
      \vspace*{-.2in}
    \item n-tuples contain a finite number of entities
      %
      \vspace*{-.2in}
    \item While the type of entity in an n-tuple may be different, not
      every entity in the n-tuple must be different
      %
  \end{itemize}
  %
\end{frame}

% Slide
%
\begin{frame}{Applying n-Tuples and Lists in Python Programs}
  %
  \begin{itemize}
    %
    \item n-tuples and lists are frequently used in Python programs
      %
      \vspace*{-.2in}
      %
    \item Python programs can use CSV files and databases
      %
      \vspace*{-.2in}
      %
    \item Using n-tuples inside of a Python program:
      %
      \begin{itemize}
        %
        \item {\bf Q1}: What is the difference between a list and a tuple?
          %
        \item {\bf Q2}: How do you read a CSV file from the file system?
          %
        \item {\bf Q3}: How do you parse a CSV file encoded in a string?
          %
        \item {\bf Q4}: How do you handle the intricacies of real-world CSV
          files?
          %
        \item {\bf Q5}: How can a Python program connect to a SQLite database?
          %
      \end{itemize}
      %
      \vspace*{-.2in}
      %
    \item What are the ways in which the  mathematical concept of an n-tuple
      connects to a wide variety of practical applications in computing,
      business, and science?
      %
  \end{itemize}
  %
\end{frame}

\end{document}
