\documentclass[14pt,aspectratio=169]{beamer}

\usepackage{pgfpages}
\usepackage{fancyvrb}
\usepackage{pgfplots}

\usepackage{minted}
\usemintedstyle{tango}

\usepackage{amsfonts}

\usepackage{moresize}
\usepackage{anyfontsize}

\usepackage{tikz}
\usetikzlibrary{arrows,shapes}
\usetikzlibrary{arrows.meta}

\tikzstyle{process}=[rectangle, draw, thick, text width=5em, text centered, minimum height=2.5em, fill=gray!40]
\tikzstyle{entity}=[rounded rectangle, draw, thick, text width=5em, text centered, minimum height=1.5em, fill=gray!40]

\usetheme{auriga}
\usecolortheme{auriga}

\setbeamercolor{background canvas}{bg=lightgray}

% define some colors for a consistent theme across slides

\definecolor{red}{RGB}{181, 23, 0}
\definecolor{blue}{RGB}{0, 118, 186}
\definecolor{gray}{RGB}{146, 146, 146}

\title{Discrete Structures: \\ Programming with \\ Finite Sets in Sympy}

\author{{\bf Gregory M. Kapfhammer}}

\institute[shortinst]{{\bf Department of Computer Science, Allegheny College}}

\begin{document}

{
  \setbeamercolor{page number in head/foot}{fg=background canvas.bg}
  \begin{frame}
    \titlepage
  \end{frame}
}

%% Slide
%
\begin{frame}{Technical Question}
  %
  \hspace*{.5in}
  %
  \begin{minipage}{4.3in}
    %
    \vspace*{.1in}
    %
    \begin{center}
      %
      {\large How do I use the implementation of a finite set in Sympy to
      create Python programs that calculate probabilities?}
      %
    \end{center}
    %
  \end{minipage}
  %
  \vspace{2ex}
  %
  \begin{center}
    %
    \small Let's explore how to use finite sets in Sympy to implement programs
    that calculate the probability of an event! Let's better understand
    how discrete structures help us to characterize random events!
    %
  \end{center}
  %
\end{frame}

% Slide
%
\begin{frame}{Using Mathematical Sets in Python Programs}
  %
  \begin{itemize}
    %
    \item Set theory is important to many areas of mathematics
      %
      \vspace*{-.15in}
      %
    \item The {\tt Sympy} package gives an implementation of finite sets
      %
      \begin{itemize}
        %
        \item Remember, sets are ``containers'' for other elements
        %
        \item The sets in Sympy are finite sets, called {\tt FiniteSet}
        %
        \item These sets have the same properties as built-in sets
        %
        \item {\tt FiniteSet} has a few features not provided by {\tt set}
        %
        \item Sets can container other objects like sets and tuples
        %
        \item A subset of a set contains a portion of the set
        %
      \end{itemize}
      %
      \vspace*{-.2in}
      %
    \item How do these properties of sets make it easier to implement Python
      programs? How are they different than lists or tuples or generator
      expressions?
      %
  \end{itemize}
  %
\end{frame}

% Slide
%
\begin{frame}[fragile]
  \frametitle{Using Python to Create a Set from a List}
  \normalsize
  \begin{minipage}{6in}
    \vspace*{.25in}
    \begin{minted}[mathescape, numbersep=5pt, fontsize=\large]{python}
x = set(['pencil', 'paper', 'pen',
         'pencil', 'wallet', 'pen'])

print("Set defined with a list:")
print(x)
    \end{minted}
  \end{minipage}
  \vspace*{.25in}
  \begin{center}
    %
    \normalsize \noindent What is the output of this program? \\
    \normalsize \noindent What are the properties of a set in Python? \\
    \normalsize \noindent How do we define the order of the set's elements? \\
    %
  \end{center}
  %
\end{frame}

% Slide
%
\begin{frame}[fragile]
  \frametitle{Using Python to Create a Set from a Tuple}
  \normalsize
  \begin{minipage}{6in}
    \vspace*{.25in}
    \begin{minted}[mathescape, numbersep=5pt, fontsize=\large]{python}
x = set(('pencil', 'paper', 'pen',
         'pencil', 'wallet', 'pen'))

print("Set defined with a tuple:")
print(x)
    \end{minted}
  \end{minipage}
  \vspace*{.25in}
  \begin{center}
    %
    \normalsize \noindent What is the output of this program? \\
    \normalsize \noindent Does the use of a set or a tuple influence? \\
    \normalsize \noindent What types of elements can we store in a set? \\
    %
  \end{center}
  %
\end{frame}

% Slide
%
\begin{frame}[fragile]
  \frametitle{Output of a Program that Creates Sets}
  \normalsize
  \begin{minipage}{6in}
    \vspace*{.25in}
    \begin{minted}[mathescape, numbersep=5pt, fontsize=\large]{text}

Set defined with a list:
{'paper', 'pencil', 'pen', 'wallet'}

Set defined with a tuple:
{'paper', 'pencil', 'pen', 'wallet'}

    \end{minted}
  \end{minipage}
  \vspace*{.25in}
  \begin{center}
    %
    \normalsize \noindent The sets do not store duplicate values \\
    \normalsize \noindent Creation from either a set or a tuple is the same \\
    \normalsize \noindent The contents of a set are displayed in lexicographic
    order \\
    %
  \end{center}
  %
\end{frame}

% Slide
%
\begin{frame}[fragile]
  \frametitle{Python's Sets Store Immutable Elements}
  \normalsize
  \begin{minipage}{6in}
    \vspace*{.25in}
    \begin{minted}[mathescape, numbersep=5pt, fontsize=\large]{python}

x = {53, 'pencil',
     (1, 1, 2, 3, 5), 3.14159}

print("Set with multiple types:")
print(x)

    \end{minted}
  \end{minipage}
  \vspace*{.05in}
  \begin{center}
    %
    \normalsize \noindent Note the different style for creation \\
    \normalsize \noindent The data types in this set are different \\
    \normalsize \noindent All of the variables in this set are immutable \\
    \normalsize \noindent How will this program display the elements of the set? \\
    %
  \end{center}
  %
\end{frame}

% Slide
%
\begin{frame}[fragile]
  \frametitle{Python's Sets Cannot Store Mutable Data}
  \normalsize
  \begin{minipage}{6in}
    \vspace*{.25in}
    \begin{minted}[mathescape, numbersep=5pt, fontsize=\large]{python}
list = [53, 'pencil',
        (1, 1, 2, 3, 5), 3.14159]
x = {list}
    \end{minted}
    %
    \vspace*{.15in}
    %
    \begin{minted}[mathescape, numbersep=5pt, fontsize=\large]{text}
Traceback (most recent call last):
  File "set-element-types.py", line 11,
       in <module>
    x = {list}
TypeError: unhashable type: 'list'
    \end{minted}
  \end{minipage}
  \vspace*{.05in}
  \begin{center}
    %
    \normalsize \noindent Note the different style for creation \\
    %
  \end{center}
  %
\end{frame}

% Slide
%
\begin{frame}{Creating Sets with Mathematical Notation}
  %
  \begin{itemize}
    %
    \item Explicit definition of a set: $S = \{1, 2, 3\}$
      %
      \vspace*{-.15in}
      %
    \item Definition of a set with a property:\\ $\{ n \; | \; 0 < n < 100
      \;\mbox{and}\; n \;\mbox{is odd} \}$
      %
      \vspace*{-.15in}
      %
    \item $\mathbb{N}$ is the set of natural numbers
      %
      \vspace*{-.15in}
      %
    \item $\mathbb{Z}$ is the set of integer numbers
      %
      \vspace*{-.15in}
      %
    \item $\mathbb{R}$ is the set of real numbers
      %
      \vspace*{-.15in}
      %
    \item $\mathbb{C}$ is the set of complex numbers
      %
      \vspace*{-.15in}
      %
    \item You can also define sets by using different set operators!
      %
  \end{itemize}
  %
\end{frame}

% Slide
%
\begin{frame}{Mathematical Operations with Sets}
  %
  \begin{itemize}
    %
    \item Set membership: $S = \{1, 2, 3\}$ such that $1 \in S$ and $5 \notin S$
      %
      \vspace*{-.15in}
      %
    \item Proper subset: $S = \{1, 2, 3\}$ and thus $\{1\} \subset S$
      %
      \vspace*{-.15in}
      %
    \item Subset: $S = \{1, 2, 3\}$ and thus $\{1\} \subseteq S$ and $\{1, 2, 3\} \subseteq S$
      %
      \vspace*{-.15in}
      %
    \item Set union: $S_1 \cup S_2$ contains elements in either $S_1$ and $S_2$
      %
      \vspace*{-.15in}
      %
    \item Set intersection: $S_1 \cap S_2$ is the elements in both $S_1$ and $S_2$
      %
      \vspace*{-.15in}
      %
    \item Set difference: $S_1 - S_2$ is the elements in $S_1$ but not in $S_2$
      %
      \vspace*{-.15in}
      %
    \item Union and intersection are associative and commutative!
      %
  \end{itemize}
  %
\end{frame}

% Slide
%
\begin{frame}{Differences Between Math and Programming}
  %
  \begin{itemize}
    %
    \item Programmers cannot use sets like mathematicians do!
      %
      \vspace*{-.15in}
      %
    \item Python programs cannot store an infinite set
      %
      \vspace*{-.15in}
      %
    \item Finite sets must fit into a computer's finite memory
      %
      \vspace*{-.15in}
      %
    \item Programs need a procedure for constructing the set
      %
      \vspace*{-.15in}
      %
    \item Different programming languages and packages have other restrictions.
      For instance, recall that Python programs cannot create sets that contain
      mutable elements like lists!
      %
      \vspace*{-.15in}
      %
    \item So, what are the benefits of using sets in Python programs?
      %
  \end{itemize}
  %
\end{frame}

% Slide
%
\begin{frame}[fragile]
  \frametitle{Using the Set Union Operator in Python}
  \normalsize
  \begin{minipage}{6in}
    \vspace*{.25in}
    \begin{minted}[mathescape, numbersep=5pt, fontsize=\large]{python}
a = {1, 2, 3, 4}
b = {2, 3, 4, 5}
c = {3, 4, 5, 6}
d = {4, 5, 6, 7}

print(a.union(b, c, d))
print(a | b | c | d)
    \end{minted}
  \end{minipage}
  \vspace*{.05in}
  \begin{center}
    %
    \normalsize \noindent Sets in Python provide useful operations like {\tt union}! \\
    %
  \end{center}
  %
\end{frame}

% Slide
%
\begin{frame}[fragile]
  \frametitle{Using the Set Intersection Operator in Python}
  \normalsize
  \begin{minipage}{6in}
    \vspace*{.25in}
    \begin{minted}[mathescape, numbersep=5pt, fontsize=\large]{python}
a = {1, 2, 3, 4}
b = {2, 3, 4, 5}
c = {3, 4, 5, 6}
d = {4, 5, 6, 7}

print(a.intersection(b, c, d))
print(a & b & c & d)
    \end{minted}
  \end{minipage}
  \vspace*{.05in}
  \begin{center}
    %
    \normalsize \noindent Sets in Python provide useful operations like {\tt
    intersection}! \\
    %
  \end{center}
  %
\end{frame}

% Slide
%
\begin{frame}[fragile]
  \frametitle{Using the Set Difference Operator in Python}
  \normalsize
  \begin{minipage}{6in}
    \vspace*{.25in}
    \begin{minted}[mathescape, numbersep=5pt, fontsize=\large]{python}
a = {1, 2, 3, 4}
b = {2, 3, 4, 5}
c = {3, 4, 5, 6}
d = {4, 5, 6, 7}

print(a.difference(b, c))
print(a - b - c)
    \end{minted}
  \end{minipage}
  \vspace*{.05in}
  \begin{center}
    %
    \normalsize \noindent Sets in Python provide useful operations like {\tt
    difference}! \\
    %
  \end{center}
  %
\end{frame}

% Slide
%
\begin{frame}[fragile]
  \frametitle{Output of the Set Operators in Python}
  \normalsize
  \begin{minipage}{6in}
    \vspace*{.25in}
    \begin{minted}[mathescape, numbersep=5pt, fontsize=\normalsize]{text}
Union of the sets:
{1, 2, 3, 4, 5, 6, 7}
{1, 2, 3, 4, 5, 6, 7}

Intersection of the sets:
{4}
{4}

Difference of the sets:
{1}
{1}
    \end{minted}
  \end{minipage}
  \vspace*{.05in}
  \begin{center}
    %
    \normalsize \noindent Sets in Python provide useful operations like {\tt
    difference}! \\
    %
  \end{center}
  %
\end{frame}

% Slide
%
\begin{frame}[fragile]
  \frametitle{Set Comprehensions in Python: Odd Positives}
  \normalsize
  \begin{minipage}{6in}
    \vspace*{.25in}
    \begin{minted}[mathescape, numbersep=5pt, fontsize=\large]{python}
odd_positives = {n for n in range(100)
                 if n % 2 == 1}

for odd_positive in odd_positives:
    print(odd_positive)

print(list(odd_positives))
    \end{minted}
  \end{minipage}
  \vspace*{.05in}
  \begin{center}
    %
    \normalsize \noindent Modular arithmetic helps us determine when a number is
    odd\\
    %
  \end{center}
  %
\end{frame}

% Slide
%
\begin{frame}[fragile]
  \frametitle{Set Comprehensions in Python: Even Positives}
  \normalsize
  \begin{minipage}{6in}
    \vspace*{.25in}
    \begin{minted}[mathescape, numbersep=5pt, fontsize=\large]{python}
even_positives = {n for n in range(100)
                  if n % 2 == 0}

for even_positive in even_positives:
    print(even_positive)

print(list(even_positives))
    \end{minted}
  \end{minipage}
  \vspace*{.05in}
  \begin{center}
    %
    \normalsize \noindent Modular arithmetic helps us determine when a number is
    even\\
    %
  \end{center}
  %
\end{frame}

% Slide
%
\begin{frame}[fragile]
  \frametitle{Boolean Logic and Sets: Using ``Logical Or''}
  \normalsize
  \begin{minipage}{6in}
    \vspace*{.25in}
    \begin{minted}[mathescape, numbersep=5pt, fontsize=\large]{python}
odd_positives_two =
        {n for n in range(20)
         if n % 2 == 1 or n == 2}

for value in odd_positives_two:
    print(value)

print(list(odd_positives_two))
    \end{minted}
    %
  \end{minipage}
  %
\end{frame}

% Slide
%
\begin{frame}[fragile]
  \frametitle{Output of the Program Using ``Logical Or''}
  \normalsize
  \begin{minipage}{6in}
    \vspace*{.15in}
    \begin{minted}[mathescape, numbersep=5pt, fontsize=\large]{text}
1
2
3
[...]
19

[1, 2, 3, 5, 7, 9, 11, 13, 15, 17, 19]
    \end{minted}
    %
  \end{minipage}
  \vspace*{.05in}
  \begin{center}
    %
    \normalsize \noindent Either of the conditions must be true for logical
    {\tt or} operator!\\
    %
  \end{center}
  %
  %
\end{frame}

% Slide
%
\begin{frame}[fragile]
  \frametitle{Boolean Logic and Sets: Using ``Logical And''}
  \normalsize
  \begin{minipage}{6in}
    \vspace*{.25in}
    \begin{minted}[mathescape, numbersep=5pt, fontsize=\large]{python}
even_positives_by_four =
     {n for n in range(20)
      if n % 2 == 0 and n % 4 == 0}

for value in even_positives_by_four:
    print(value)

print(list(even_positives_by_four))
    \end{minted}
    %
  \end{minipage}
  %
\end{frame}

% Slide
%
\begin{frame}[fragile]
  \frametitle{Output of the Program Using ``Logical And''}
  \normalsize
  \begin{minipage}{6in}
    \vspace*{.15in}
    \begin{minted}[mathescape, numbersep=5pt, fontsize=\large]{text}
0
4
8
12
16

[0, 4, 8, 12, 16]
    \end{minted}
    %
  \end{minipage}
  \vspace*{.05in}
  \begin{center}
    %
    \normalsize \noindent Both of the conditions must be true for logical
    {\tt and} operator!\\
    %
  \end{center}
  %
  %
\end{frame}

% Slide
%
\begin{frame}{Using Sets in Python Programs}
  %
  \begin{itemize}
    %
    \item Sets are a discrete structure with many practical benefits!
      %
      \vspace*{-.2in}
      %
    \item Sets have built-in operations that make programming easy
      %
      \vspace*{-.2in}
      %
    \item Using sets and Boolean logic in Python programs:
      %
      \begin{itemize}
        %
        \item {\bf Q1}: What are the characteristics of a set?
          %
        \item {\bf Q2}: What are the built-in operations provided by a set?
          %
        \item {\bf Q3}: How can you connect sets in math and programming?
          %
        \item {\bf Q4}: How does Boolean logic help us describe a set?
          %
        \item {\bf Q5}: How does the {\tt sympy} package support set programming?
          %
      \end{itemize}
      %
      \vspace*{-.2in}
      %
    \item Refer to \url{https://realpython.com/python-sets/} for more details
      about how to program in Python with sets!
      %
  \end{itemize}
  %
\end{frame}

\end{document}
