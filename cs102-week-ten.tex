\documentclass[14pt,aspectratio=169]{beamer}

\usepackage{pgfpages}
\usepackage{fancyvrb}
\usepackage{pgfplots}

\usepackage{minted}
\usemintedstyle{tango}

\usepackage{amsfonts}

\usepackage{moresize}
\usepackage{anyfontsize}

\usepackage{tikz}
\usetikzlibrary{arrows,shapes}
\usetikzlibrary{arrows.meta}

\tikzstyle{process}=[rectangle, draw, thick, text width=5em, text centered, minimum height=2.5em, fill=gray!40]
\tikzstyle{entity}=[rounded rectangle, draw, thick, text width=5em, text centered, minimum height=1.5em, fill=gray!40]

\usetheme{auriga}
\usecolortheme{auriga}

\setbeamercolor{background canvas}{bg=lightgray}

% define some colors for a consistent theme across slides

\definecolor{red}{RGB}{181, 23, 0}
\definecolor{blue}{RGB}{0, 118, 186}
\definecolor{gray}{RGB}{146, 146, 146}

\title{Discrete Structures: \\ Programming with \\ Finite Sets in Sympy}

\author{{\bf Gregory M. Kapfhammer}}

\institute[shortinst]{{\bf Department of Computer Science, Allegheny College}}

\begin{document}

{
  \setbeamercolor{page number in head/foot}{fg=background canvas.bg}
  \begin{frame}
    \titlepage
  \end{frame}
}

%% Slide
%
\begin{frame}{Technical Question}
  %
  \hspace*{.5in}
  %
  \begin{minipage}{4.3in}
    %
    \vspace*{.1in}
    %
    \begin{center}
      %
      {\large How do I use the implementation of a finite set in Sympy to
      create Python programs that calculate probabilities?}
      %
    \end{center}
    %
  \end{minipage}
  %
  \vspace{2ex}
  %
  \begin{center}
    %
    \small Let's explore how to use finite sets in Sympy to implement programs
    that calculate the probability of an event! Let's better understand
    how discrete structures help us to characterize random events!
    %
  \end{center}
  %
\end{frame}

% Slide
%
\begin{frame}{Using Mathematical Sets in Python Programs}
  %
  \begin{itemize}
    %
    \item Set theory is useful in mathematics and computer science
      %
      \vspace*{-.15in}
      %
    \item The {\tt Sympy} package gives an implementation of finite sets
      %
      \begin{itemize}
        %
        \item Remember, sets are ``containers'' for other elements
        %
        \item The sets in Sympy are finite sets, called {\tt FiniteSet}
        %
        \item These sets have the same properties as built-in sets
        %
        \item {\tt FiniteSet} has a few features not provided by {\tt set}
        %
        \item A probability is the likelihood that an event will occur
        %
        \item We can use either {\tt set} or {\tt FiniteSet} to study
          probabilities
        %
      \end{itemize}
      %
      \vspace*{-.2in}
      %
    \item This week we will investigate probability after learning about an
      alternative approach to creating and using sets
      %
  \end{itemize}
  %
\end{frame}

% Slide
%
\begin{frame}[fragile]
  \frametitle{Using Python to Create a Finite Set}
  \normalsize
  \begin{minipage}{6in}
    \vspace*{.25in}
    \begin{minted}[mathescape, numbersep=5pt, fontsize=\large]{python}
empty_set = FiniteSet()
print(empty_set)

finite_set = FiniteSet(2, 4, 6, 8, 10)
print(finite_set)
    \end{minted}
  \end{minipage}
  \vspace*{.25in}
  \begin{center}
    %
    \normalsize \noindent What is the output of this program? \\
    \normalsize \noindent Are there other ways to create a {\tt FiniteSet}? \\
    \normalsize \noindent What operations can we perform with a {\tt FiniteSet}? \\
    %
  \end{center}
  %
\end{frame}

% Slide
%
\begin{frame}[fragile]
  \frametitle{Using Sympy to Create a Set from a List or Tuple}
  \normalsize
  \begin{minipage}{6in}
    \vspace*{.25in}
    \begin{minted}[mathescape, numbersep=5pt, fontsize=\large]{python}
list = [2, 4, 6, 8, 10]
finite_set = FiniteSet(*list)
print(finite_set)

tuple = (2, 4, 6, 8, 10)
finite_set = FiniteSet(*tuple)
print(finite_set)
    \end{minted}
  \end{minipage}
  \vspace*{.05in}
  \begin{center}
    %
    \normalsize \noindent What is the purpose of the {\tt *} in this program? \\
    %
  \end{center}
  %
\end{frame}

% Slide
%
\begin{frame}[fragile]
  \frametitle{Output of a Program that Creates Finite Sets}
  \normalsize
  \begin{minipage}{6in}
    \vspace*{.25in}
    \begin{minted}[mathescape, numbersep=5pt, fontsize=\normalsize]{text}
Explicit FiniteSet:
  FiniteSet(2, 4, 6, 8, 10)

Empty FiniteSet:
  EmptySet

FiniteSet from Tuple:
  FiniteSet(2, 4, 6, 8, 10)

FiniteSet Containing Tuple:
  FiniteSet((2, 4, 6, 8, 10))
    \end{minted}
  \end{minipage}
  %
\end{frame}

% Slide
%
\begin{frame}{Creating Sets with Mathematical Notation}
  %
  \begin{itemize}
    %
    \item Explicit definition of a set: $S = \{1, 2, 3\}$
      %
      \vspace*{-.15in}
      %
    \item Definition of a set with a property:\\ $\{ n \; | \; 0 < n < 100
      \;\mbox{and}\; n \;\mbox{is odd} \}$
      %
      \vspace*{-.15in}
      %
    \item $\mathbb{N}$ is the set of natural numbers
      %
      \vspace*{-.15in}
      %
    \item $\mathbb{Z}$ is the set of integer numbers
      %
      \vspace*{-.15in}
      %
    \item $\mathbb{R}$ is the set of real numbers
      %
      \vspace*{-.15in}
      %
    \item $\mathbb{C}$ is the set of complex numbers
      %
      \vspace*{-.15in}
      %
    \item Which of these sets are {\bf finite}? Which of them are {\bf infinite}?
      %
  \end{itemize}
  %
\end{frame}

% Slide
%
\begin{frame}{Mathematical Operations with Sets}
  %
  \begin{itemize}
    %
    \item Set membership: $S = \{1, 2, 3\}$ such that $1 \in S$ and $5 \notin S$
      %
      \vspace*{-.15in}
      %
    \item Proper subset: $S = \{1, 2, 3\}$ and thus $\{1\} \subset S$
      %
      \vspace*{-.15in}
      %
    \item Subset: $S = \{1, 2, 3\}$ and thus $\{1\} \subseteq S$ and $\{1, 2, 3\} \subseteq S$
      %
      \vspace*{-.15in}
      %
    \item Set union: $S_1 \cup S_2$ contains elements in either $S_1$ and $S_2$
      %
      \vspace*{-.15in}
      %
    \item Set intersection: $S_1 \cap S_2$ is the elements in both $S_1$ and $S_2$
      %
      \vspace*{-.15in}
      %
    \item Set difference: $S_1 - S_2$ is the elements in $S_1$ but not in $S_2$
      %
      \vspace*{-.15in}
      %
    \item Can we perform these operations using {\tt FiniteSet}?
      %
  \end{itemize}
  %
\end{frame}

% Slide
%
\begin{frame}{Differences Between Math and Programming}
  %
  \begin{itemize}
    %
    \item Programmers cannot use sets like mathematicians do!
      %
      \vspace*{-.15in}
      %
    \item Python programs cannot store an infinite set
      %
      \vspace*{-.15in}
      %
    \item Finite sets must fit into a computer's finite memory
      %
      \vspace*{-.15in}
      %
    \item Programs need a procedure for constructing the set
      %
      \vspace*{-.15in}
      %
    \item Different programming languages and packages have other restrictions.
      For instance, recall that Python programs cannot create sets that contain
      mutable elements like lists!
      %
      \vspace*{-.15in}
      %
    \item So, what are the benefits of using {\tt FiniteSet} in Sympy?
      %
  \end{itemize}
  %
\end{frame}

% Slide
%
\begin{frame}[fragile]
  \frametitle{Set Repetition and Iteration with Finite Sets}
  \normalsize
  \begin{minipage}{6in}
    \vspace*{.25in}
    \begin{minted}[mathescape, numbersep=5pt, fontsize=\large]{python}
from sympy import FiniteSet

list = [1, 2, 3, 2]
finite_set = FiniteSet(*list)
print(finite_set)

for element in finite_set:
    print(element)
    \end{minted}
  \end{minipage}
  %
\end{frame}

% Slide
%
\begin{frame}[fragile]
  \frametitle{Checking Subset Relationships with Finite Sets}
  \normalsize
  \begin{minipage}{6in}
    \vspace*{.25in}
    \begin{minted}[mathescape, numbersep=5pt, fontsize=\large]{python}
one = FiniteSet(1, 2, 3)
two = FiniteSet(1, 2, 3)

subset = one.is_proper_subset(two)
print(subset)

subset = two.is_proper_subset(one)
print(subset)
    \end{minted}
  \end{minipage}
  %
\end{frame}

% Slide
%
\begin{frame}[fragile]
  \frametitle{Checking Subset Relationships with Finite Sets}
  \normalsize
  \begin{minipage}{6in}
    \vspace*{.25in}
    \begin{minted}[mathescape, numbersep=5pt, fontsize=\large]{python}
one = FiniteSet(1, 2, 3)
three = FiniteSet(1, 2, 3, 4)

subset = one.is_proper_subset(three)
print(subset)

subset = three.is_proper_subset(one)
print(subset)
    \end{minted}
  \end{minipage}
  %
\end{frame}

% Slide
%
\begin{frame}{Using Sets in Python Programs}
  %
  \begin{itemize}
    %
    \item Sets are a discrete structure with many practical benefits!
      %
      \vspace*{-.2in}
      %
    \item Sets have built-in operations that make programming easy
      %
      \vspace*{-.2in}
      %
    \item Using sets and Boolean logic in Python programs:
      %
      \begin{itemize}
        %
        \item {\bf Q1}: What are the characteristics of a set?
          %
        \item {\bf Q2}: What are the built-in operations provided by a set?
          %
        \item {\bf Q3}: How can you connect sets in math and programming?
          %
        \item {\bf Q4}: How does Boolean logic help us describe a set?
          %
        \item {\bf Q5}: How does the {\tt sympy} package support set programming?
          %
      \end{itemize}
      %
      \vspace*{-.2in}
      %
    \item Refer to \url{https://realpython.com/python-sets/} for more details
      about how to program in Python with sets!
      %
  \end{itemize}
  %
\end{frame}

\end{document}
