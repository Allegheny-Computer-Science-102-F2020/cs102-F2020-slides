\documentclass[14pt,aspectratio=169]{beamer}

\usepackage{pgfpages}
\usepackage{fancyvrb}
\usepackage{pgfplots}

\usepackage{minted}
\usemintedstyle{tango}

\usepackage{amsfonts}

\usepackage{moresize}
\usepackage{anyfontsize}

\usepackage{tikz}
\usetikzlibrary{arrows,shapes}
\usetikzlibrary{arrows.meta}

\tikzstyle{process}=[rectangle, draw, thick, text width=5em, text centered, minimum height=2.5em, fill=gray!40]
\tikzstyle{entity}=[rounded rectangle, draw, thick, text width=5em, text centered, minimum height=1.5em, fill=gray!40]

\usetheme{auriga}
\usecolortheme{auriga}

\setbeamercolor{background canvas}{bg=lightgray}

% define some colors for a consistent theme across slides
\definecolor{red}{RGB}{181, 23, 0}
\definecolor{blue}{RGB}{0, 118, 186}
\definecolor{gray}{RGB}{146, 146, 146}

\title{Discrete Structures: \\ Ordered Pairs, Tuples, and Monoids in Python}

\author{{\bf Gregory M. Kapfhammer}}

\institute[shortinst]{{\bf Department of Computer Science, Allegheny College}}

\begin{document}

{
  \setbeamercolor{page number in head/foot}{fg=background canvas.bg}
  \begin{frame}
    \titlepage
  \end{frame}
}

%% Slide
%
\begin{frame}{Technical Question}
  %
  \hspace*{.25in}
  \begin{minipage}{5in}
  \begin{center}
    %
    {\large How do I employ the mathematical concepts of order pairs, n-tuples,
      monoids, and sequences to implement efficient Python programs that use
    higher-order functions with a clearly specified behavior?}
    %
  \end{center}
  \end{minipage}
  %
  \vspace{2ex}
  %
  \begin{center}
    %
    \small Let's learn how to translate concepts from discrete mathematics to
    implement efficient Python programs that are rigorously specified!
    %
  \end{center}
  %
\end{frame}

% Slide
%
\begin{frame}{Understanding Ordered Pairs}
  %
  \begin{itemize}
    %
    \item Mathematical concepts yield predictable programs
      %
      \vspace*{-.15in}
      %
    \item Understanding an ``ordered pair''
      %
      \begin{itemize}
        %
        \item {\bf Pair}: grouping of two entities
          %
        \item {\bf Ordered}: order of entities matters
          %
        \item {\bf Ordered Pair}: grouping of two entities for which order
          matters
          %
        \item {\bf Coordinate on Earth}: latitude and longitude are an ordered
          pair
          %
        \item {\bf Complex Numbers}: real and imaginary parts are an ordered
          pair
          %
        \item An ordered pair is not the same as a set of two elements!
          %
      \end{itemize}
      %
      \vspace*{-.2in}
      %
    \item If we have ordered pairs of entities, can we generalize to
      an ordered grouping beyond two entities? How?
      %
  \end{itemize}
  %
\end{frame}

% Slide
%
\begin{frame}[fragile]
  \frametitle{Practical Applications of Ordered Pairs}
  % \hspace*{-.15in}
  \begin{minipage}{6in}
    \vspace*{.25in}
    \begin{minted}[mathescape, numbersep=5pt, fontsize=\large]{text}
(40.758896° N, -73.985130° E)
  Times Square
(60.1699° N, 24.9384° E)
  Helsinki, Finland
(90.0° N, 0.0° E)
  North Pole
    \end{minted}
  \end{minipage}
  \vspace*{.05in}
  \begin{center}
    %
    \normalsize \noindent Latitude and longitude provide a ``global address''
    for a location\\
    %
    \normalsize \noindent Why does the order matter for these pairs?\\
    %
  \end{center}
  %
\end{frame}

% Slide
%
\begin{frame}{Generalizing Ordered Pairs to n-Tuples}
  %
  \begin{itemize}
    %
    \item Mathematical concepts yield predictable programs
      %
      \vspace*{-.15in}
      %
    \item Understanding an ``ordered pair''
      %
      \begin{itemize}
        %
        \item {\bf Pair}: grouping of two entities
          %
        \item {\bf Ordered}: order of entities matters
          %
        \item {\bf Ordered Pair}: grouping of two entities for which order
          matters
          %
        \item {\bf Coordinate on Earth}: latitude and longitude are an ordered
          pair
          %
        \item {\bf Complex Numbers}: real and imaginary parts are an ordered
          pair
          %
        \item An ordered pair is not the same as a set of two elements!
          %
      \end{itemize}
      %
      \vspace*{-.2in}
      %
    \item If we have ordered pairs of entities, can we generalize to
      an ordered grouping beyond two entities? How?
      %
  \end{itemize}
  %
\end{frame}


% Slide
%
\begin{frame}{Implementing and Debugging Python Functions}
  %
  \begin{itemize}
    %
    \item Use debugging statements to grasp a function's behavior!
      %
      \vspace*{-.15in}
      %
    \item Python functions to perform statistical analysis of data
      %
      \begin{itemize}
        %
        \item {\bf Q1}: How do you compute the median of a list of numbers?
          %
        \item {\bf Q2}: How do you compute the mode of a list of numbers?
          %
        \item {\bf Q3}: How do you compute a frequency table of a list of
          numbers?
          %
        \item {\bf Q4}: How do you compute the range of a list of numbers?
          %
        \item {\bf Q5}: How do you compute the variance and standard deviation?
          %
      \end{itemize}
      %
      \vspace*{-.2in}
      %
    \item Can you translate the mathematical descriptions of these summary
      statistics to Python programs? Can you ensure their correctness? Can you
      follow industry best practices?
      %
  \end{itemize}
  %
\end{frame}

\end{document}
