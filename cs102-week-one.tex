\documentclass[14pt,aspectratio=169]{beamer}
\usepackage{pgfpages}
\usepackage{fancyvrb}
\usepackage{tikz}
\usepackage{pgfplots}

\usetheme{auriga}
\usecolortheme{auriga}

% define some colors for a consistent theme across slides
\definecolor{red}{RGB}{181, 23, 0}
\definecolor{blue}{RGB}{0, 118, 186}
\definecolor{gray}{RGB}{146, 146, 146}

\title{Discrete Structures: \\ Python Programming Tools}

\author{Gregory M. Kapfhammer}

\institute[shortinst]{Department of Computer Science, Allegheny College}

\begin{document}

{
  \setbeamercolor{page number in head/foot}{fg=background canvas.bg}
  \begin{frame}
    \titlepage
  \end{frame}
}

%% Slide

\begin{frame}{Technical Question}

  \begin{center}
    %
    {\large How do I install and use the industry-standard programming tools that will
    help me to rigorously explore discrete structures with the Python
  programming language?}
    %
  \end{center}

  \vspace{2ex}

  \begin{center}
    %
    \small Let's learn more about version control, text editors, Docker,
    the Markdown language for technical writing, and the Python programming
    language!
    %
  \end{center}

\end{frame}

%% Slide

\begin{frame}{A slide title}

  \begin{itemize}
    \item A bulleted item
    \item Another item
      \begin{itemize}
        \item With sub-bullets
        \item And another, with some \textbf{bold} text
      \end{itemize}
    \item And another, at the top level, with \textit{italic} text
  \end{itemize}

  \note{
    Here's a note for this slide.
  }

\end{frame}

\end{document}
