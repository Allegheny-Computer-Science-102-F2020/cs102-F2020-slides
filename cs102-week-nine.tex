\documentclass[14pt,aspectratio=169]{beamer}

\usepackage{pgfpages}
\usepackage{fancyvrb}
\usepackage{pgfplots}

\usepackage{minted}
\usemintedstyle{tango}

\usepackage{amsfonts}

\usepackage{moresize}
\usepackage{anyfontsize}

\usepackage{tikz}
\usetikzlibrary{arrows,shapes}
\usetikzlibrary{arrows.meta}

\tikzstyle{process}=[rectangle, draw, thick, text width=5em, text centered, minimum height=2.5em, fill=gray!40]
\tikzstyle{entity}=[rounded rectangle, draw, thick, text width=5em, text centered, minimum height=1.5em, fill=gray!40]

\usetheme{auriga}
\usecolortheme{auriga}

\setbeamercolor{background canvas}{bg=lightgray}

% define some colors for a consistent theme across slides

\definecolor{red}{RGB}{181, 23, 0}
\definecolor{blue}{RGB}{0, 118, 186}
\definecolor{gray}{RGB}{146, 146, 146}

\title{Discrete Structures: \\ Programming with \\ Sets in Python}

\author{{\bf Gregory M. Kapfhammer}}

\institute[shortinst]{{\bf Department of Computer Science, Allegheny College}}

\begin{document}

{
  \setbeamercolor{page number in head/foot}{fg=background canvas.bg}
  \begin{frame}
    \titlepage
  \end{frame}
}

%% Slide
%
\begin{frame}{Technical Question}
  %
  \hspace*{.5in}
  %
  \begin{minipage}{4.3in}
    %
    \vspace*{.1in}
    %
    \begin{center}
      %
      {\large How do I use the mathematical concepts of sets and Boolean logic
      to design Python programs that are easier to implement and understand?}
      %
    \end{center}
    %
  \end{minipage}
  %
  \vspace{2ex}
  %
  \begin{center}
    %
    \small Let's explore how to use sets implement efficient and effective
    programs that are easy to understand! Let's better understand how the choice
    of a discrete structure influences the readability of a Python program!
    %
  \end{center}
  %
\end{frame}

% Slide
%
\begin{frame}{Using Mathematical Sets in Python Programs}
  %
  \begin{itemize}
    %
    \item Set theory is important to many areas of mathematics
      %
      \vspace*{-.15in}
      %
    \item Concentrate on the set theory useful in programming
      %
      \begin{itemize}
        %
        \item Sets are ``containers'' for other elements
        %
        \item Sets do not contain duplicate values
        %
        \item Set elements are not stored in a specific order
        %
        \item The universal set is the set of all elements
        %
        \item Sets can container other objects like sets and tuples
        %
        \item A subset of a set contains a portion of the set
        %
      \end{itemize}
      %
      \vspace*{-.2in}
      %
    \item How do these properties of sets make it easier to implement Python
      programs? How are they different than lists or tuples or generator
      expressions?
      %
  \end{itemize}
  %
\end{frame}

% Slide
%
\begin{frame}[fragile]
  \frametitle{Using Python to Create a Set from a List}
  \normalsize
  \begin{minipage}{6in}
    \vspace*{.25in}
    \begin{minted}[mathescape, numbersep=5pt, fontsize=\large]{python}
x = set(['pencil', 'paper', 'pen',
         'pencil', 'wallet', 'pen'])

print("Set defined with a list:")
print(x)
    \end{minted}
  \end{minipage}
  \vspace*{.25in}
  \begin{center}
    %
    \normalsize \noindent What is the output of this program? \\
    \normalsize \noindent What are the properties of a set in Python? \\
    \normalsize \noindent How do we define the order of the set's elements? \\
    %
  \end{center}
  %
\end{frame}

% Slide
%
\begin{frame}[fragile]
  \frametitle{Using Python to Create a Set from a Tuple}
  \normalsize
  \begin{minipage}{6in}
    \vspace*{.25in}
    \begin{minted}[mathescape, numbersep=5pt, fontsize=\large]{python}
x = set(('pencil', 'paper', 'pen',
         'pencil', 'wallet', 'pen'))

print("Set defined with a tuple:")
print(x)
    \end{minted}
  \end{minipage}
  \vspace*{.25in}
  \begin{center}
    %
    \normalsize \noindent What is the output of this program? \\
    \normalsize \noindent Does the use of a set or a tuple influence? \\
    \normalsize \noindent What types of elements can we store in a set? \\
    %
  \end{center}
  %
\end{frame}

% Slide
%
\begin{frame}[fragile]
  \frametitle{Output of a Program that Creates Sets}
  \normalsize
  \begin{minipage}{6in}
    \vspace*{.25in}
    \begin{minted}[mathescape, numbersep=5pt, fontsize=\large]{text}

Set defined with a list:
{'paper', 'pencil', 'pen', 'wallet'}

Set defined with a tuple:
{'paper', 'pencil', 'pen', 'wallet'}

    \end{minted}
  \end{minipage}
  \vspace*{.25in}
  \begin{center}
    %
    \normalsize \noindent The sets do not store duplicate values \\
    \normalsize \noindent Creation from either a set or a tuple is the same \\
    \normalsize \noindent The contents of a set are displayed in lexicographic
    order \\
    %
  \end{center}
  %
\end{frame}

% Slide
%
\begin{frame}[fragile]
  \frametitle{Python's Sets Store Immutable Elements}
  \normalsize
  \begin{minipage}{6in}
    \vspace*{.25in}
    \begin{minted}[mathescape, numbersep=5pt, fontsize=\large]{python}

x = {53, 'pencil',
     (1, 1, 2, 3, 5), 3.14159}

print("Set with multiple types:")
print(x)

    \end{minted}
  \end{minipage}
  \vspace*{.05in}
  \begin{center}
    %
    \normalsize \noindent Note the different style for creation \\
    \normalsize \noindent The data types in this set are different \\
    \normalsize \noindent All of the variables in this set are immutable \\
    \normalsize \noindent How will this program display the elements of the set? \\
    %
  \end{center}
  %
\end{frame}

% Slide
%
\begin{frame}[fragile]
  \frametitle{Python's Sets Cannot Store Mutable Data}
  \normalsize
  \begin{minipage}{6in}
    \vspace*{.25in}
    \begin{minted}[mathescape, numbersep=5pt, fontsize=\large]{python}
list = [53, 'pencil',
        (1, 1, 2, 3, 5), 3.14159]
x = {list}
    \end{minted}
    %
    \vspace*{.15in}
    %
    \begin{minted}[mathescape, numbersep=5pt, fontsize=\large]{text}
Traceback (most recent call last):
  File "set-element-types.py", line 11,
       in <module>
    x = {list}
TypeError: unhashable type: 'list'
    \end{minted}
  \end{minipage}
  \vspace*{.05in}
  \begin{center}
    %
    \normalsize \noindent Note the different style for creation \\
    %
  \end{center}
  %
\end{frame}

% Slide
%
\begin{frame}{Creating Sets with Mathematical Notation}
  %
  \begin{itemize}
    %
    \item Explicit definition of a set: $S = \{1, 2, 3\}$
      %
      \vspace*{-.15in}
      %
    \item Definition of a set with a property:\\ $\{ n \; | \; 0 < n < 100
      \;\mbox{and}\; n \;\mbox{is odd} \}$
      %
      \vspace*{-.15in}
      %
    \item $\mathbb{N}$ is the set of natural numbers
      %
      \vspace*{-.15in}
      %
    \item $\mathbb{Z}$ is the set of integer numbers
      %
      \vspace*{-.15in}
      %
    \item $\mathbb{R}$ is the set of real numbers
      %
      \vspace*{-.15in}
      %
    \item $\mathbb{C}$ is the set of complex numbers
      %
      \vspace*{-.15in}
      %
    \item You can also define sets by using different set operators!
      %
  \end{itemize}
  %
\end{frame}

% Slide
%
\begin{frame}{Using Streams in Python Programs}
  %
  \begin{itemize}
    %
    \item Python programs frequently generate streams of values
      %
      \vspace*{-.2in}
      %
    \item Using streams enables the program to be memory efficient
      %
      \vspace*{-.2in}
      %
    \item Using tuples, lists, and streams in Python programs:
      %
      \begin{itemize}
        %
        \item {\bf Q1}: What is the difference between a list and a tuple?
          %
        \item {\bf Q2}: What is the difference between a list and stream?
          %
        \item {\bf Q3}: What is the difference between a tuple and a stream?
          %
        \item {\bf Q4}: How can you concatenate streams into a new stream?
          %
        \item {\bf Q5}: How do streams aid distributed programming in Python?
          %
      \end{itemize}
      %
      \vspace*{-.2in}
      %
    \item Dynamically generated streams support memory efficiency and
      flexibility in Python programs. Also, the stream is a monoid under the
      identify of the empty stream!
      %
  \end{itemize}
  %
\end{frame}

\end{document}
