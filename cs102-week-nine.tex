\documentclass[14pt,aspectratio=169]{beamer}

\usepackage{pgfpages}
\usepackage{fancyvrb}
\usepackage{pgfplots}

\usepackage{minted}
\usemintedstyle{tango}

\usepackage{amsfonts}

\usepackage{moresize}
\usepackage{anyfontsize}

\usepackage{tikz}
\usetikzlibrary{arrows,shapes}
\usetikzlibrary{arrows.meta}

\tikzstyle{process}=[rectangle, draw, thick, text width=5em, text centered, minimum height=2.5em, fill=gray!40]
\tikzstyle{entity}=[rounded rectangle, draw, thick, text width=5em, text centered, minimum height=1.5em, fill=gray!40]

\usetheme{auriga}
\usecolortheme{auriga}

\setbeamercolor{background canvas}{bg=lightgray}

% define some colors for a consistent theme across slides

\definecolor{red}{RGB}{181, 23, 0}
\definecolor{blue}{RGB}{0, 118, 186}
\definecolor{gray}{RGB}{146, 146, 146}

\title{Discrete Structures: \\ Programming with \\ Sets in Python}

\author{{\bf Gregory M. Kapfhammer}}

\institute[shortinst]{{\bf Department of Computer Science, Allegheny College}}

\begin{document}

{
  \setbeamercolor{page number in head/foot}{fg=background canvas.bg}
  \begin{frame}
    \titlepage
  \end{frame}
}

%% Slide
%
\begin{frame}{Technical Question}
  %
  \hspace*{.5in}
  %
  \begin{minipage}{4.3in}
    %
    \vspace*{.1in}
    %
    \begin{center}
      %
      {\large How do I use the mathematical concepts of sets and Boolean logic
      to design Python programs that are easier to implement and understand?}
      %
    \end{center}
    %
  \end{minipage}
  %
  \vspace{2ex}
  %
  \begin{center}
    %
    \small Let's explore how to use sets implement efficient and effective
    programs that are easy to understand! Let's better understand how the choice
    of a discrete structure influences the readability of a Python program!
    %
  \end{center}
  %
\end{frame}

% slide
%
\begin{frame}{Dynamically Generated Streams}
  %
  \begin{itemize}
    %
    \item Sequences and streams are evident in Python programs!
      %
      \vspace*{-.15in}
      %
    \item Static sequences and dynamically generated streams:
      %
      \begin{itemize}
        %
        \item A list is a static sequence that exists as a complete data
          structure
        %
        \item An file input stream appears dynamically over time when read
        %
        \item Let's distinguish between static sequences and dynamic streams
        %
        \item Example: streams are generated by iterators and range objects
        %
      \end{itemize}
      %
      \vspace*{-.2in}
      %
    \item What are the benefits of having a dynamic stream whose values are not
      known until they are generated and returned? How does the memory
      consumption of a Python program influence our choice of a discrete
      structure?
      %
  \end{itemize}
  %
\end{frame}

% Slide
%
\begin{frame}[fragile]
  \frametitle{File Input Involves the Use of Streams}
  \normalsize
  \hspace*{-.65in}
  \begin{minipage}{6in}
    \vspace*{.25in}
    \begin{minted}[mathescape, numbersep=5pt, fontsize=\large]{python}
    file = open("emails")
    for line in file:
      name, email = line.split(",")
      if name == "John Davis":
        print(email)
    \end{minted}
  \end{minipage}
  \vspace*{.25in}
  \begin{center}
    %
    \normalsize \noindent The file is a sequence of characters \\
    \normalsize \noindent A character is a sequence of numbers \\
    \normalsize \noindent Does the entire file exist in the computer's
    memory? \\
    %
  \end{center}
  %
\end{frame}

% Slide
%
\begin{frame}{Using Streams in Python Programs}
  %
  \begin{itemize}
    %
    \item Python programs frequently generate streams of values
      %
      \vspace*{-.2in}
      %
    \item Using streams enables the program to be memory efficient
      %
      \vspace*{-.2in}
      %
    \item Using tuples, lists, and streams in Python programs:
      %
      \begin{itemize}
        %
        \item {\bf Q1}: What is the difference between a list and a tuple?
          %
        \item {\bf Q2}: What is the difference between a list and stream?
          %
        \item {\bf Q3}: What is the difference between a tuple and a stream?
          %
        \item {\bf Q4}: How can you concatenate streams into a new stream?
          %
        \item {\bf Q5}: How do streams aid distributed programming in Python?
          %
      \end{itemize}
      %
      \vspace*{-.2in}
      %
    \item Dynamically generated streams support memory efficiency and
      flexibility in Python programs. Also, the stream is a monoid under the
      identify of the empty stream!
      %
  \end{itemize}
  %
\end{frame}

\end{document}
