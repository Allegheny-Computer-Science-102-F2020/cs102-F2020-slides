\documentclass[14pt,aspectratio=169]{beamer}

\usepackage{pgfpages}
\usepackage{fancyvrb}
\usepackage{pgfplots}

\usepackage{minted}
\usemintedstyle{tango}

\usepackage{amsfonts}

\usepackage{moresize}
\usepackage{anyfontsize}

\usepackage{tikz}
\usetikzlibrary{arrows,shapes}
\usetikzlibrary{arrows.meta}

\tikzstyle{process}=[rectangle, draw, thick, text width=5em, text centered, minimum height=2.5em, fill=gray!40]
\tikzstyle{entity}=[rounded rectangle, draw, thick, text width=5em, text centered, minimum height=1.5em, fill=gray!40]

\usetheme{auriga}
\usecolortheme{auriga}

\setbeamercolor{background canvas}{bg=lightgray}

% define some colors for a consistent theme across slides
\definecolor{red}{RGB}{181, 23, 0}
\definecolor{blue}{RGB}{0, 118, 186}
\definecolor{gray}{RGB}{146, 146, 146}

\title{Discrete Structures: \\ Debugging Functions Implemented in Python}

\author{{\bf Gregory M. Kapfhammer}}

\institute[shortinst]{{\bf Department of Computer Science, Allegheny College}}

\begin{document}

{
  \setbeamercolor{page number in head/foot}{fg=background canvas.bg}
  \begin{frame}
    \titlepage
  \end{frame}
}

%% Slide
%
\begin{frame}{Technical Question}
  %
  \begin{center}
    %
    {\large How do I use debugging statements to better understand the behavior
    of functions that use iteration and recursion to perform mathematical
  operations such as computing the factorial sequence, the square of a number,
and the mean and median of a sequence of numbers?}
    %
  \end{center}
  %
  \vspace{1ex}
  %
  \begin{center}
    %
    \small Let's learn how to use Python {\tt print} statements to debug
    functions that perform mathematical and statistical computations!
    %
  \end{center}
  %
\end{frame}

% Slide
%
\begin{frame}{Debugging Python Functions}
  %
  \begin{itemize}
    %
    \item Intuitively read the functions to grasp their behavior
      %
      \vspace*{-.15in}
      %
    \item Key components of the Python functions
      %
      \begin{itemize}
        %
        \item Definition of the function
          %
        \item Parameter(s) that serve as the input
          %
        \item Body that performs a computation
          %
        \item Function return value(s) that produce output
          %
        \item Invocation of the function
          %
        \item Collecting the output of the function
          %
        \item Test case(s) for the function
          %
      \end{itemize}
      %
      \vspace*{-.2in}
      %
    \item Learn ways add {\tt print} statements to make debugging output,
      helping to understand a function's behavior
      %
  \end{itemize}
  %
\end{frame}

% Slide
%
\begin{frame}{Implementing and Debugging Python Functions}
  %
  \begin{itemize}
    %
    \item Use debugging statements to grasp a function's behavior!
      %
      \vspace*{-.15in}
      %
    \item Python functions to perform statistical analysis of data
      %
      \begin{itemize}
        %
        \item {\bf Q1}: How do you compute the median of a list of numbers?
          %
        \item {\bf Q2}: How do you compute the mode of a list of numbers?
          %
        \item {\bf Q3}: How do you compute a frequency table of a list of
          numbers?
          %
        \item {\bf Q4}: How do you compute the range of a list of numbers?
          %
        \item {\bf Q5}: How do you compute the variance and standard deviation?
          %
      \end{itemize}
      %
      \vspace*{-.2in}
      %
    \item Can you translate the mathematical descriptions of these summary
      statistics to Python programs? Can you ensure their correctness? Can you
      follow industry best practices?
      %
  \end{itemize}
  %
\end{frame}

\end{document}
